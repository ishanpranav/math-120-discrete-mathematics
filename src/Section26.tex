\documentclass[12pt]{article}
\usepackage[english]{babel}
\usepackage[letterpaper,top=2cm,bottom=2cm,left=3cm,right=3cm,marginparwidth=1.75cm]{geometry}
\usepackage{amsmath}
\usepackage{amsfonts}
\usepackage{graphicx}
\DeclareMathOperator{\dom}{dom}
\DeclareMathOperator{\im}{im}
\title{MATH-UA 120 Section 26}
\author{Ishan Pranav}
\date{October 30, 2023}
\begin{document}
\maketitle
\section*{Composition of functions}
Let $A,B,$ and $C$ be sets, and let $f:A\to B$, and $g:B\to C$. Then the function $g\circ f$ is a function from $A$ to $C$ defined by
\[(g\circ f)(a)=g\left[f(a)\right]\]
where $a\in A$. The function $g\circ f$ is called the \textit{composition of $g$ and $f$}.
\section*{Identity function}
Let $A$ be a set. The \textit{identity function on $A$} is the function $\mathrm{id}_A$ with $\dom{\mathrm{id}_A}=A$, and for all $a\in A$, we have $\mathrm{id}_A(a)=a$. In other words,
\[\mathrm{id}_A=\{(a,a):a\in A\}.\]
\section*{Proposition}
Let $A,B,C,$ and $D$ be sets, and let $f:A\to B$, $g:B\to C$, and $h:C\to D$. Then
\[h\circ(g\circ f)=(h\circ g)\circ f.\]
\section*{Proposition}
Let $A$ and $B$ be sets. Let $f:A\to B$. Then
\[f\circ\mathrm{id}_A=\mathrm{id}_B\circ f=f.\]
\section*{Proposition}
Let $A$ and $B$ be sets and suppose $f:A\to B$ is an injection. Then
\[f\circ f^{-1}=\mathrm{id}_B.\]
\[f^{-1}\circ f=\mathrm{id}_A.\]
\end{document}