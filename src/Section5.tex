\documentclass[12pt]{article}
\usepackage[english]{babel}
\usepackage[letterpaper,top=2cm,bottom=2cm,left=3cm,right=3cm,marginparwidth=1.75cm]{geometry}
\usepackage{amsfonts}
\usepackage{amsmath}
\usepackage{amssymb}
\usepackage{graphicx}
\renewcommand{\labelenumi}{\alph{enumi}.}
\title{MATH-UA 120 Section 5}
\author{Ishan Pranav}
\date{September 10, 2023}
\begin{document}
\maketitle
\section*{Goldbach conjecture}
Every even integer greater than two is the sum of two primes.
\section*{Proposition 1}
The sum of two even integers is even.

We show that if $x$ and $y$ are even integers, then $x+y$ is an even integer. Let $x$ and $y$ be even integers. Since $x$ is even, we know $2 \mid x$ by definition. Likewise, since $y$ is even, $2 \mid y$. Since $2 \mid y$, we know that there is an integer $a$ such that $x=2a$ by definition. Likewise, since $2 \mid y$, there is an integer $b$ such that $y=2b$. Observe that $x+y=2+2b=2(a+b)$. Therefore there is an integer $c$ (namely, $a+b$) such that $x+y=2c$. Therefore $2|(x+y)$. Therefore $x+y$ is even.
\section*{Proposition 2}
Let $a$, $b$, and $c$ be integers. If $a \mid b$ and $b \mid c$, then $a \mid c$.

Suppose $a$, $b$, and $c$ are integers with $a \mid b$ and $b \mid c$. Since $a \mid b$, there is an integer $x$ such that $b=ax$. Likewise there is an integer $y$ such that $c=by$. Let $z=xy$. Then $az=a(xy)=(ax)y=by=c$.

Therefore there is an integer $z$ such that $c=az$. Therefore $a \mid c$.
\section*{Proposition 3}
Let $x$ be an integer and suppose $x>1$. Note that $x^3+1=(x+1)(x^2-x+1)$. Because $x$ is an integer, both $x+1$ and $x^2-x+1$ are integers. Therefore $(x+1) \mid (x^3+1)$.

Since $x>1$, we have $x+1>1+1=2>1$.

Also, $x>1$ implies that $x^2>x$, and since $x>1$, we have $x^2>1$. Multiplying both sides by $x$ again yields $x^3>x$. Adding 1 to both sides gives $x^3+1>x+1$.

Thus $x+1$ is an integer with $1<x+1<x^3+1$.

Since $x+1$ is a divisor of $x^3+1$ and $1<x+1<x^3+1$, we have that $x^3+1$ is composite.
\section*{Proposition 4}
Let $x$ be an integer. Then $x$ is even if and only if $x+1$ is odd.

Suppose $x$ is even. This means that $2 \mid x$. Hence there is an integer $a$ such that $x=2a$. Adding 1 to both sides gives $x+1=2a+1$. By the definition of \textit{odd}, $x+1$ is odd.
Suppose $x+1$ is odd. So there is an integer $b$ such that $x+1=2b+1$. Subtracting 1 from both sides gives $x=2b$. This shows that $2 \mid x$ and therefore $x$ is even.
\section*{Proposition 5}
Let $a$, $b$, $c$, and $d$ be integers. If $a \mid b$, $b \mid c$, and $c \mid d$, then $a \mid d$.

Since $a \mid b$, there is an integer $x$ such that $ax=b$.

Since $b \mid c$, there is an integer $y$ such that $by=c$.

Since $c \mid d$, there is an integer $z$ such that $cz=d$.

Note that $a(xyz)=(ax)(yz)=b(yz)=(by)z=cz=d$.

Therefore there is an integer $w=xyz$ such that $aw=d$.

Therefore $a \mid d$.
\section{The sum of two odd integers is even}
Let $x\in\mathbb{Z}$ and $y\in\mathbb{Z}$ such that $x$ is odd and $y$ is odd. Since $x$ is odd, there exists $a\in\mathbb{Z}$ such that $x=2a+1$. Since $y$ is odd, there exists $b\in\mathbb{Z}$ such that $y=2b+1$. Observe
\begin{align*}
x+y
&=(2a+1)+(2b+1)\\
&=2a+2b+2\\
&=2(a+b+1).
\end{align*}
Let $c=a+b+1\in\mathbb{Z}$. There exists $c\in\mathbb{Z}$ such that $x+y=2c$. Therefore, $x+y$ is even. $\blacksquare$
\section{The sum of an odd integer and an even integer is odd}
Let $x\in\mathbb{Z}$ and $y\in\mathbb{Z}$ such that $x$ is even and $y$ is odd. Since $x$ is even, $2 \mid x$. Thus, there exists $a\in\mathbb{Z}$ such that $x=2a$. Since $y$ is odd, there exists $b\in\mathbb{Z}$ such that $y=2b+1$. Observe
\begin{align*}
x+y
&=(2a)+(2b+1)\\
&=2(a+b)+1.
\end{align*}
Let $c=a+b\in\mathbb{Z}$. There exists $c\in\mathbb{Z}$ such that $x+y=2c+1$. Therefore, $x+y$ is odd. $\blacksquare$
\section{If $n$ is an odd integer, then $-n$ is also odd}
Let $n\in\mathbb{Z}$ such that $n$ is odd. Since $n$ is odd, there exists $a\in\mathbb{Z}$ such that $n=2a+1$. Observe
\begin{align*}
-n
&=-(2a+1)\\
&=-2a-1\\
&=2(-a)-1\\
&=2(-a-1)+1.
\end{align*}
Let $b=-a-1\in\mathbb{Z}$. There exists $b\in\mathbb{Z}$ such that $-n=2b+1$. Therefore, $-n$ is odd. $\blacksquare$
\section{The product of two even integers is even}
Let $x\in\mathbb{Z}$ and $y\in\mathbb{Z}$ such that $x$ is even and $y$ is even. Since $x$ is even, $2 \mid x$. Thus, there exists $a\in\mathbb{Z}$ such that $x=2a$. Since $y$ is even, $2 \mid y$. Thus, there exists $b\in\mathbb{Z}$ such that $y=2b$. Observe
\begin{align*}
xy
&=(2a)(2b)\\
&=2(2ab).
\end{align*}
Let $c=2ab\in\mathbb{Z}$. There exists $c\in\mathbb{Z}$ such that $xy=2c$. Thus, $2 \mid xy$. Therefore, $xy$ is even. $\blacksquare$
\section{The product of an even integer and an odd integer is even}
Let $x\in\mathbb{Z}$ and $y\in\mathbb{Z}$ such that $x$ is even and $y$ is odd. Since $x$ is even, $2 \mid x$. Thus, there exists $a\in\mathbb{Z}$ such that $x=2a$. Since $y$ is odd, there exists $b\in\mathbb{Z}$ such that $y=2b+1$. Observe
\begin{align*}
xy
&=(2a)(2b+1)\\
&=2(2ab+a).
\end{align*}
Let $c=2ab+a\in\mathbb{Z}$. There exists $c\in\mathbb{Z}$ such that $xy=2c$. Thus, $2 \mid xy$. Therefore, $xy$ is even. $\blacksquare$
\section{The product of two odd integers is odd}
Let $x\in\mathbb{Z}$ and $y\in\mathbb{Z}$ such that $x$ is odd and $y$ is odd. Since $x$ is odd, there exists $a\in\mathbb{Z}$ such that $x=2a+1$. Since $y$ is odd, there exists $b\in\mathbb{Z}$ such that $y=2b+1$. Observe
\begin{align*}
xy
&=(2a+1)(2b+1)\\
&=2a+2b+4ab+1\\
&=2(a+b+2ab)+1.
\end{align*}
Let $c=a+b+2ab\in\mathbb{Z}$. There exists $c\in\mathbb{Z}$ such that $xy=2c+1$. Therefore, $xy$ is odd. $\blacksquare$
\section{The square of an odd integer is odd}
Let $x\in\mathbb{Z}$ such that $x$ is odd. Since $x$ is odd, there exists $a\in\mathbb{Z}$ such that $x=2a+1$. Observe
\begin{align*}
x^2
&=(2a+1)^2\\
&=(2a+1)(2a+1)\\
&=4a^2+4a+1\\
&=2(2a^2+2a)+1.
\end{align*}
Let $b=2a^2+2a\in\mathbb{Z}$. There exists $b\in\mathbb{Z}$ such that $x^2=2b+1$. Therefore, $x^2$ is odd. $\blacksquare$
\section{The cube of an odd integer is odd}
Let $x\in\mathbb{Z}$ such that $x$ is odd. Since $x$ is odd, there exists $a\in\mathbb{Z}$ such that $x=2a+1$. Observe
\begin{align*}
x^3
&=(2a+1)^3\\
&=8a^3+12a^2+6a+1\\
&=2(4a^3+6a^2+3a)+1.
\end{align*}
Let $b=4a^3+6a^2+3a\in\mathbb{Z}$. There exists $b\in\mathbb{Z}$ such that $x^3=2b+1$. Therefore, $x^3$ is odd. $\blacksquare$
\section{Given $a,b,c\in\mathbb{Z}$, if $a \mid b$ and $a \mid c$, then $a \mid (b+c)$}
Let $a\in\mathbb{Z}$, $b\in\mathbb{Z}$, and $c\in\mathbb{Z}$ such that $a \mid b$ and $a \mid c$. Since $a \mid b$, there exists $x\in\mathbb{Z}$ such that $b=ax$. Since $a \mid c$, there exists $y\in\mathbb{Z}$ such that $c=ay$. Observe
\begin{align*}
b+c
&=ax+ay\\
&=a(x+y).
\end{align*}
Let $z=x+y\in\mathbb{Z}$. There exists $z\in\mathbb{Z}$ such that $b+c=az$. Therefore, $a \mid (b+c).\,\blacksquare$
\section{Given $a,b,c\in\mathbb{Z}$, if $a \mid b$, then $a \mid bc$}
Let $a\in\mathbb{Z}$, $b\in\mathbb{Z}$, and $c\in\mathbb{Z}$ such that $a \mid b$. Since $a \mid b$, there exists $x\in\mathbb{Z}$ such that $b=ax$. Note $bc=acx$. Let $y=cx\in\mathbb{Z}$. There exists $y\in\mathbb{Z}$ such that $bc=ay$. Therefore, $a \mid bc.\,\blacksquare$
\section{Given $a,b,d,x,y\in\mathbb{Z}$, if $d \mid a$ and $d \mid b$, then $d \mid (ax+by)$}
Let $a\in\mathbb{Z}$, $b\in\mathbb{Z}$, $d\in\mathbb{Z}$, $x\in\mathbb{Z}$, and $y\in\mathbb{Z}$ such that $d \mid a$ and $d \mid b$. Since $d \mid a$, there exists $c\in\mathbb{Z}$ such that $a=cd$. Since $d \mid b$, there exists $z\in\mathbb{Z}$ such that $b=dz$. Observe
\begin{align*}
ax+by
&=(cd)(x)+(dz)(z)\\
&=d(cx+yz).
\end{align*}
Let $r=cx+yz\in\mathbb{Z}$. There exists $r\in\mathbb{Z}$ such that $ax+by=dr$. Therefore, $d \mid (ax+by).\,\blacksquare$
\section{Given $a,b,c,d\in\mathbb{Z}$, if $a \mid b$ and $c \mid d$, then $ac \mid bd$}
Let $a\in\mathbb{Z}$, $b\in\mathbb{Z}$, $c\in\mathbb{Z}$, and $d\in\mathbb{Z}$ such that $a \mid b$ and $c \mid d$. Since $a \mid b$, there exists $x\in\mathbb{Z}$ such that $b=ax$. Since $c \mid d$, there exists $y\in\mathbb{Z}$ such that $d=cy$. Note $bd=acxy$. Let $z=xy\in\mathbb{Z}$. There exists $z\in\mathbb{Z}$ such that $bd=acz$. Therefore $ac \mid bd.\,\blacksquare$
\section{Given $x\in\mathbb{Z}$, $x$ is odd if and only if $x+1$ is even}
Let $x\in\mathbb{Z}$.

First, we will prove that if $x+1$ is even, then $x$ is odd. Suppose $x+1$ is even. Since $x+1$ is even, $2 \mid (x+1)$. Thus, there exists $y\in\mathbb{Z}$ such that $x+1=2y$. Observe
\begin{align*}
x+1&=2y\\
x&=2y-1\\
x&=2(y-1)+1.
\end{align*}
Let $z=y-1\in\mathbb{Z}$. There exists $z\in\mathbb{Z}$ such that $x=2z+1$. Therefore, $x$ is odd if $x+1$ is even.

Next, we will prove that if $x$ is odd, then $x+1$ is even. Suppose $x$ is odd. Since $x$ is odd, there exists $a\in\mathbb{Z}$ such that $x=2a+1$. Observe
\begin{align*}
x&=2a+1\\
x+1&=2a+2\\
x+1&=2(a+1).
\end{align*}
Let $b=a+1\in\mathbb{Z}$. There exists $b\in\mathbb{Z}$ such that $x+1=2b$. Therefore, $x+1$ is even if $x$ is odd.

We conclude that $x$ is odd if and only if $x+1$ is even. $\blacksquare$
\section{$x$ is odd if and only if $\exists\,b\in\mathbb{Z}$ such that $x=2b-1$}
Let $x\in\mathbb{Z}$. 

First, we will prove that if there exists $b\in\mathbb{Z}$ such that $x=2b-1$, then $x$ is odd. Suppose there exists $b\in\mathbb{Z}$ such that $x=2b-1$. Observe
\[x=2b-1=2(b-1)+1.\]
Let $a=b-1\in\mathbb{Z}$. There exists $a\in\mathbb{Z}$ such that $x=2a+1$. Therefore, $x$ is odd.

Next, we will prove that if $x$ is odd, then there exists $b\in\mathbb{Z}$ such that $x=2b-1$. Suppose $x$ is odd. Since $x$ is odd, there exists $c\in\mathbb{Z}$ such that $x=2c+1$. Observe
\[x=2c+1=2(c+1)-1.\]
Let $b=c+1\in\mathbb{Z}$. Therefore, there exists $b\in\mathbb{Z}$ such that $x=2b-1$.

We conclude that $x$ is odd if and only if there exists $b\in\mathbb{Z}$ such that $x=2b-1.\,\blacksquare$
\section{Given $x\in\mathbb{Z}$, $0 \mid x$ if and only if $x=0$}
Let $x\in\mathbb{Z}$.

First, we will prove that if $x=0$, then $0 \mid x$. We want to find $n\in\mathbb{Z}$ such that $x=0n$. Note $x=0n=0$ for all integers $n$ (including 0). Let $n=0$. Therefore $0 \mid x$.

Next, we will prove that if $0 \mid x$, then $x=0$. Since $0 \mid x$, there exists $n\in\mathbb{Z}$ such that $x=0n=0$. Therefore $x=0$.

We conclude that $0 \mid x$ if and only if $x=0.\,\blacksquare$
\end{document}
