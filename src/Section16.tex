\documentclass[12pt]{article}
\usepackage[english]{babel}
\usepackage[letterpaper,top=2cm,bottom=2cm,left=3cm,right=3cm,marginparwidth=1.75cm]{geometry}
\usepackage{amsmath}
\usepackage{amsfonts}
\usepackage{graphicx}
\title{MATH-UA 120 Section 16}
\author{Ishan Pranav}
\date{October 3, 2023}
\begin{document}
\maketitle
\section*{Partition}
Let $A$ be a set. A \textit{partition of} (or \textit{on}) $A$ is a set of nonempty, pairwise disjoint sets whose union is $A$.
\section*{Part}
A partition is a set of sets; each member of a partition is a subset of $A$. The members of the partition are called \textit{parts}.
The parts of a partition are nonempty. The empty set is never a part of a partition.
The parts of a partition are pairwise disjoint. No two parts of a partition may have an element in common. The union of the parts is the original set.
\section*{Corollary 15}
Let $R$ be an equivalence relation on a set $A$. The equivalence classes of $R$ form a partition of the set $A$.
\section*{Counting equivalence classes}
Let $R$ be an equivalence relation on a finite set $A$. If all the equivalence classes of $R$ have the same size, $m$, then the number of equivalence classes is $\frac{|A|}{m}$.
\end{document}