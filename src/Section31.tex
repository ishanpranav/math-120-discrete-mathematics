\documentclass[12pt]{article}
\usepackage[english]{babel}
\usepackage[letterpaper,top=2cm,bottom=2cm,left=3cm,right=3cm,marginparwidth=1.75cm]{geometry}
\usepackage{amsmath}
\usepackage{amsfonts}
\usepackage{graphicx}
\DeclareMathOperator{\dom}{dom}
\DeclareMathOperator{\im}{im}
\title{MATH-UA 120 Section 31}
\author{Ishan Pranav}
\date{November 5, 2023}
\begin{document}
\maketitle
\section*{Event}
Let $(S,P)$ be a sample space. An \textit{event} $A$ is a subset of $S$ ($A\subseteq S$). The probability of an event $A$, denoted $P(A)$ is
\[P(A)=\sum_{a\in A}{P(a)}.\]
\section*{Proposition}
Let $A$ and $B$ be events in a sample space $(S,P)$. Then
\[P(A)+P(B)=P(A\cup B)+P(A\cap B).\]
\section*{Proposition}
Let $A$ and $B$ be events. If $A\cap B=\emptyset$, then $P(A\cup B)=P(A)+P(B).$
\section*{Proposition}
Let $A$ and $B$ be events. Then $P(A\cup B)\leq P(A)+P(B)$.
\section*{Proposition}
Let $(S,P)$ be a sample space. Then $P(S)=1.$
\section*{Proposition}
Note $P(\emptyset)=0.$
\section*{Proposition}
Let $(S,P)$ be a sample space and let $A$ be an event. Then $P(\overline{A})=1-P(A)$.
\end{document}