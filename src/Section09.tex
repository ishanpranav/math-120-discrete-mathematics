\documentclass[12pt]{article}
\usepackage[english]{babel}
\usepackage[letterpaper,top=2cm,bottom=2cm,left=3cm,right=3cm,marginparwidth=1.75cm]{geometry}
\usepackage{amsmath}
\usepackage{amsfonts}
\usepackage{graphicx}
\title{MATH-UA 120 Section 9}
\author{Ishan Pranav}
\date{September 16, 2023}
\begin{document}
\maketitle
\section*{Factorial}
The quantity
\[(n)_n=n(n-1)(n-2)(n-3)\dots(n-n+1)=(n)(n-1)(n-2)\dots(1)\]
occurs frequently in mathematics and has a special name and notation; it is called $n$ \textit{factorial} and is written $n!$.

\[0!=1.\]

Formally,
\[n!=\prod_{k=1}^{n}{k}.\]
\section{}
A computer program to find the next factorion after 145: 

\begin{verbatim}
   #include <limits.h>
   #include <stdio.h>
\end{verbatim}

\verb|/**|

\verb| * |Computes the first factorion greater than the given number.

\verb| * |

\verb| * @param |min the previous factorion from which to begin the search

\verb| * @return |The next factorion, or the given \verb|min| if no such number exists.

\verb|*/|
\begin{verbatim}
   static int getFactorionAbove(int min)
   {
      for (int k = min + 1; k < INT_MAX; k++)
      {
         int sum = 0;
         int n = k;
    
         while (n)
         {
            int digit = n % 10;
            int factorial = 1;
    
            for (int i = 1; i <= digit; i++)
            {
               factorial *= i;
            }
    
            n /= 10;
            sum += factorial;
         }
    
         if (k == sum)
         {
            return k;
         }
      }
    
      return min;
   }
\end{verbatim}

\verb|/**|

\verb| * |The main entry point for the application.

\verb| * |

\verb| * @return |An exit code.

\verb|*/|

\begin{verbatim}
   int main()
   {
      printf("%d", getFactorionAbove(145));
    
      return 0;
   }
\end{verbatim}
\end{document}