\documentclass[9pt]{article}
\usepackage[margin=1in]{geometry}
\usepackage[utf8]{inputenc}
\usepackage{amsmath}
\usepackage{amssymb}
\usepackage{amsthm}
\title{Polished Proof 2}
\author{\textbf{Ishan Pranav}}
\begin{document}
\maketitle
\noindent\small{LEMMA.}\newline

\noindent\textit{Claim. }Let $A_0=\left\{\frac{1}{2},1
\right\}$. For all $n\in\mathbb{N}$, let
\[A_{n+1}=\{ab:a,b\in A_n\}\cup\left\{\frac{a+b}{2}:a,b\in A_n\right\}.\]
Let $a_n\in A_n$. Then $0\leq a_n\leq 1$.\newline

\noindent\textit{Proof. }Let $A_0=\left\{\frac{1}{2},1
\right\}$. For all $n\in\mathbb{N}$, let
$A_{n+1}=\{ab:a,b\in A_n\}\cup\left\{\frac{a+b}{2}:a,b\in A_n\right\}$.
Let $a_n\in A_n$. We will demonstrate that $0\leq a_n\leq 1$ by induction on $n$.\newline

\begin{description}
\item[Basis case.] Consider $n=0$. Then
\[A_n=A_0=\left\{\frac{1}{2},1
\right\}.\]
Note $0\leq\frac{1}{2}\leq 1$, and $0\leq 1\leq 1$. Therefore, for all $a_0\in A_0$, we have $0\leq a_0\leq 1$.
\item[Inductive hypothesis.] Let $k\in\mathbb{N}$. Consider $n=k$. Assume that for all $a_k\in A_k$, we have $0\leq a_k\leq 1$.
\item[Inductive step.] Consider $n=k+1$. We have 
\[A_{k+1}=\{ab:a,b\in A_k\}\cup\left\{\frac{a+b}{2}:a,b\in A_k\right\}.\]
Let $a_{k+1}\in A_{k+1}$. Thus $a_{k+1}\in\{ab:a,b\in A_k\}$ or $a_{k+1}\in\left\{\frac{a+b}{2}:a,b\in A_k\right\}$.\newline

Suppose $a_{k+1}\in\{ab:a,b\in A_k\}$. Then there exists $x_1,y_1\in A_k$ such that $a_{k+1}=x_1y_1$. Since $x_1\in A_k$ and $y_1\in A_k$, we have $0\leq x_1\leq 1$ and $0\leq y_1\leq 1$ by the inductive hypothesis. Since $x_1\geq 0$ and $y_1\geq 0$, we have $x_1y_1\geq 0$. Since $x_1\geq 0$, $y_1\geq 0$, $x_1\leq 1$, and $y_1\leq 1$, we have $x_1y_1\leq 1$. Thus $0\leq x_1y_1\leq 1$. Therefore $0\leq a_{k+1}\leq 1$.\newline

Suppose $a_{k+1}\in\left\{\frac{a+b}{2}:a,b\in A_k\right\}$. Then there exists $x_2,y_2\in A_k$ such that $a_{k+1}=\frac{x_2+y_2}{2}$. Since $x_2\in A_k$ and $y_2\in A_k$, we have $0\leq x_2\leq 1$ and $0\leq y_2\leq 1$ by the inductive hypothesis. 
Since $x_2\geq 0$ and $y_2\geq 0$, we have $x_2+y_2\geq 0$. Thus $\frac{x_2+y_2}{2}\geq 0$. Since $x_2\leq 1$ and $y_2\leq 1$, we have $\frac{x_2}{2}\leq\frac{1}{2}$ and $\frac{y_2}{2}\leq\frac{1}{2}$. Note $\left(\frac{x_2}{2}+\frac{y_2}{2}\right)\leq\left(\frac{1}{2}+\frac{1}{2}\right)$. So $\frac{x_2+y_2}{2}\leq 1$. Thus $0\leq\frac{x_2+y_2}{2}\leq 1$. Therefore $0\leq a_{k+1}\leq 1$.\newline

In all cases, for all $a_{k+1}\in A_k$, we have $0\leq a_{k+1}\leq 1$, thus completing the inductive step.\newline

\end{description}

\noindent Hence, for all $n\in\mathbb{N}$, for all $a_n\in A_n$, we have $0\leq a_n\leq 1$.\newline

~\newline

\noindent\small{PROPOSITION.}\newline

\noindent\textit{Claim. }Let $A_0=\left\{\frac{1}{2},1
\right\}$. For all $n\in\mathbb{N}$, let
\[A_{n+1}=\{ab:a,b\in A_n\}\cup\left\{\frac{a+b}{2}:a,b\in A_n\right\}.\]
Let
\[A=\bigcup_{j=0}^{\infty}{A_j}.\]
If $x\in A$, then $0\leq x\leq 1$.\newline

\noindent\textit{Proof. }Let $A_0=\left\{\frac{1}{2},1
\right\}$. For all $n\in\mathbb{N}$, let $A_{n+1}=\{ab:a,b\in A_n\}\cup\left\{\frac{a+b}{2}:a,b\in A_n\right\}$. Let $A=\bigcup_{j=0}^{\infty}{A_j}$. Let $x\in A$. Of course,
\[x\in(A_0\cup A_1\cup A_2\cup\dots).\]
So there exists $j\in\mathbb{N}$ such that $x\in A_j$. Since $j\in\mathbb{N}$ and $x\in A_j$, we have $0\leq x\leq 1$ by lemma.\newline

\noindent Hence if $x\in A$, then $0\leq x\leq 1.~~\square$
\end{document}