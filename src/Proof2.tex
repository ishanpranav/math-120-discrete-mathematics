\documentclass[9pt]{article}
\usepackage[margin=1in]{geometry}
\usepackage[utf8]{inputenc}
\usepackage{amsmath}
\usepackage{amssymb}
\usepackage{amsthm}
\title{Polished Proof 2}
\author{\textbf{Ishan Pranav}}
\begin{document}
\maketitle
\noindent\small{LEMMA.}\newline

\noindent\textit{Claim. }Let $A_0=\left\{\frac{1}{2},1
\right\}$. For all $n\in\mathbb{N}$, let
\[A_{n+1}=\{ab:a,b\in A_n\}\cup\left\{\frac{a+b}{2}:a,b\in A_n\right\}.\]
Let $a_n\in A_n$. Then $0\leq a_n\leq 1$.\newline

\noindent\textit{Proof. }Let $A_0=\left\{\frac{1}{2},1
\right\}$. For all $n\in\mathbb{N}$, let
$A_{n+1}=\{ab:a,b\in A_n\}\cup\left\{\frac{a+b}{2}:a,b\in A_n\right\}$.
Let $a_n\in A_n$. We will demonstrate that $0\leq a_n\leq 1$ by induction on $n$.\newline

\begin{description}
\item[Basis case.] Consider $n=0$. Then
\[A_n=A_0=\left\{\frac{1}{2},1
\right\}.\]
Note $0\leq\frac{1}{2}\leq 1$, and $0\leq 1\leq 1$. Therefore, for all $a_0\in A_0$, we have $0\leq a_0\leq 1$.
\item[Inductive hypothesis.] Let $k\in\mathbb{N}$. Consider $n=k$. Assume that for all $a_k\in A_k$, we have $0\leq a_k\leq 1$.
\item[Inductive step.] Consider $n=k+1$. We have 
\[A_{k+1}=\{ab:a,b\in A_k\}\cup\left\{\frac{a+b}{2}:a,b\in A_k\right\}.\]
Let $a_{k+1}\in A_{k+1}$. Thus $a_{k+1}\in\{ab:a,b\in A_k\}$ or $a_{k+1}\in\left\{\frac{a+b}{2}:a,b\in A_k\right\}$.\newline

Suppose $a_{k+1}\in\{ab:a,b\in A_k\}$. Since $a\in A_k$ and $b\in A_k$, we have $0\leq a\leq 1$ and $0\leq b\leq 1$ by the inductive hypothesis. Since $a\geq 0$ and $b\geq 0$, we have $ab\geq 0$. Since $a\geq 0$, $b\geq 0$, $a\leq 1$, and $b\leq 1$, we have $ab\leq 1$. Thus $0\leq ab\leq 1$. Of course, $a_{k+1}=ab$. Therefore, $0\leq a_{k+1}\leq 1$.\newline

Suppose $\left\{\frac{a+b}{2}:a,b\in A_k\right\}$. Since $a\in A_k$ and $b\in A_k$, we have $0\leq a\leq 1$ and $0\leq b\leq 1$ by the inductive hypothesis. 
Since $a\geq 0$ and $b\geq 0$, we have $a+b\geq 0$. Thus $\frac{a+b}{2}\geq 0$. Since $a\leq 1$ and $b\leq 1$, we have $\frac{a}{2}\leq\frac{1}{2}$ and $\frac{b}{2}\leq\frac{1}{2}$. Note $\left(\frac{a}{2}+\frac{b}{2}\right)\leq\left(\frac{1}{2}+\frac{1}{2}\right)$. So $\frac{a+b}{2}\leq 1$. Thus $0\leq\frac{a+b}{2}\leq 1$. Of course $a_{k+1}=\frac{a+b}{2}$. Therefore, $0\leq a_{k+1}\leq 1$.\newline

In all cases, for all $a_{k+1}\in A_k$, we have $0\leq a_{k+1}\leq 1$, thus completing the inductive step.\newline

\end{description}

\noindent Hence, for all $n\in\mathbb{N}$, for all $a_n\in A_n$, we have $0\leq a_n\leq 1$.\newline

~\newline

\noindent\small{PROPOSITION.}\newline

\noindent\textit{Claim. }Let $A_0=\left\{\frac{1}{2},1
\right\}$. For all $n\in\mathbb{N}$, let
\[A_{n+1}=\{ab:a,b\in A_n\}\cup\left\{\frac{a+b}{2}:a,b\in A_n\right\}.\]
Let
\[A=\bigcup_{j=0}^{\infty}{A_j}.\]
If $x\in A$, then $0\leq x\leq 1$.\newline

\noindent\textit{Proof. }Let $A_0=\left\{\frac{1}{2},1
\right\}$. For all $n\in\mathbb{N}$, let $A_{n+1}=\{ab:a,b\in A_n\}\cup\left\{\frac{a+b}{2}:a,b\in A_n\right\}$. Let $A=\bigcup_{j=0}^{\infty}{A_j}$. Let $x\in A$. Of course,
\[x\in(A_0\cup A_1\cup A_2\cup\dots).\]
So there exists $j\in\mathbb{N}$ such that $x\in A_j$. Since $j\in\mathbb{N}$ and $x\in A_j$, we have $0\leq x\leq 1$ by lemma.\newline

\noindent Hence if $x\in A$, then $0\leq x\leq 1.~~\square$
\section*{Reflection}
I found this proof very approachable. I'm a programmer first, not really a mathematician, so I approach these problems by breaking them up into smaller ``subroutines'' \textit{per se}. As a result, I prefer to use lemmas when possible. In this case, I found that isolating the inductive step and working through it on paper (as we did in class) made the problem digestible. After establishing and proving the lemma, the proof followed so neatly that it was almost self-evident. 

Before attempting this problem, I actually thought through the other problem and immediately noticed the trick: The proof would require a demonstration of Gauss's sum of consecutive natural numbers formula. However, I decided to ``pass'' on this proof because it relied so heavily on English-language explanations to interpret the story about the rocks, and I find it easier to work with more precise mathematical notation. 

I like this problem because I really enjoy induction proofs---the ``domino'' logic is so neat. I find the process of breaking down a complex problem into simple parts very satisfying. Looking back at the proof, I think the only negative was how much redundancy the lemma introduced. I had to restate the data in the lemma's claim, in the proposition's claim, and in the proofs of the lemma and proposition themselves. Perhaps rolling everything into one proposition would make it ``short and sweet,'' but I think with all the similarly named variables, readers could benefit from having everything compartmentalized.
\end{document}