\documentclass[12pt]{article}
\usepackage[english]{babel}
\usepackage[letterpaper,top=2cm,bottom=2cm,left=3cm,right=3cm,marginparwidth=1.75cm]{geometry}
\usepackage{amsmath}
\usepackage{amsfonts}
\usepackage{graphicx}
\title{MATH-UA 120 Section 15}
\author{Ishan Pranav}
\date{October 1, 2023}
\begin{document}
\maketitle
\section*{Equivalence relation}
Let $R$ be a relation on a set $A$. We say $R$ is an \textit{equivalence relation} provided it is reflexive, symmetric, and transitive.
\section*{Congruence modulo $n$}
Let $n$ be a positive integer. We say that integers $x$ and $y$ are \textit{congruent modulo} $n$, and we write $x\equiv y~(\mathrm{mod}~n)$, provided $n\mid (x-y)$. In other words, $x$ and $y$ are congruent modulo $n$ if and only if $x$ and $y$ differ by a multiple of $n$. We often abbreviate the word \textit{modulo} to just \textit{mod}. Any two integers are congruent modulo 1. Two numbers are congruent modulo 2 if and only if they are both even or both odd.
\section*{Equivalence class}
Let $R$ be an equivalence relation on a set $A$ and let $a\in A$. The \textit{equivalence class of} $a$, denoted $[a]$, is the set of all elements of $A$ related (by $R$) to $a$; that is,
\[[a]=\{x\in A:x~R~a\}.\].
\section*{Corollary 15}
Let $R$ be an equivalence relation on set $A$. The equivalence classes of $R$ are nonempty pairwise disjoint subsets of $A$ whose union is $A$.
\end{document}