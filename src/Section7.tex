\documentclass[12pt]{article}
\usepackage[english]{babel}
\usepackage[letterpaper,top=2cm,bottom=2cm,left=3cm,right=3cm,marginparwidth=1.75cm]{geometry}
\usepackage{amsfonts}
\usepackage{amsmath}
\usepackage{amssymb}
\usepackage{graphicx}
\renewcommand{\labelenumi}{\alph{enumi}.}
\title{MATH-UA 120 Section 7}
\author{Ishan Pranav}
\date{September 12, 2023}
\begin{document}
\maketitle
\section*{And}
The operation \textit{and}, denoted $\land$, is defined:

\[\top\land\top=\top,\]

\[\top\land\bot=\bot,\]

\[\bot\land\top=\bot,\]

\[\bot\land\bot=\bot.\]
\section*{Or}
The operation \textit{or}, denoted $\lor$, is defined:

\[\top\lor\top=\top,\]

\[\top\lor\bot=\top,\]

\[\bot\lor\top=\top,\]

\[\bot\lor\bot=\bot.\]
\section*{Not}
The operation \textit{not}, denoted $\lnot$, is defined:

\[\lnot\top=\bot,\]

\[\lnot\bot=\top.\]
\section*{Proposition 6}
The Boolean expressions $\lnot(x\land y)$ and $(\lnot x)\lor(\lnot y)$ are logically equivalent.

\begin{center}
\begin{tabular}{c|c||c|c||c|c|c}
$x$&$y$&$x\land y$&$\lnot(x\land y)$&$\lnot x$&$\lnot y$&$(\lnot x)\lor(\lnot y)$\\
$\top$&$\top$&$\top$&$\bot$&$\bot$&$\bot$&$\bot$\\
$\top$&$\bot$&$\bot$&$\top$&$\bot$&$\top$&$\top$\\
$\bot$&$\top$&$\bot$&$\top$&$\top$&$\bot$&$\top$\\
$\bot$&$\top$&$\bot$&$\top$&$\top$&$\bot$&$\top$
\end{tabular}
\end{center}
\section*{Commutative property of \textit{and}}
\[x\land y=y\land x.\]
\section*{Commutative property of \textit{or}}
\[x\lor y=y\lor x.\]
\section*{Associative property of \textit{and}}
\[(x\land y)\land z=x\land(y\land z).\]
\section*{Associative property of \textit{or}}
\[(x\lor y)\lor z=x\lor(y\lor z).\]
\section*{Identity property of \textit{and}}
\[x\land\top=x.\]
\section*{Identity property of \textit{or}}
\[x\lor\bot=x.\]
\section*{\textit{And} property}
\[x\land x=x.\]
\section*{\textit{Or} property}
\[x\lor x=x.\]
\section*{\textit{Not} property}
\[\lnot(\lnot x)=x.\]
\section*{Distributive property of \textit{and}}
\[x\land(y\lor z)=(x\land y)\lor(x\land z).\]
\section*{Distributive property of \textit{or}}
\[x\lor(y\land z)=(x\lor y)\land(x\lor z).\]
\section*{\textit{False} property}
\[x\land(\lnot x)=\bot.\]
\section*{\textit{True} property}
\[x\lor(\lnot x)=\top.\]
\section*{De Morgan's laws}
\[\lnot(x\land y)=(\lnot x)\lor(\lnot y),\]

\[\lnot(x\lor y)=(\lnot x)\land(\lnot y).\]
\section*{Material conditional}
The \textit{material conditional} operation (also called an \textit{if-then} or \textit{implication}), denoted $\rightarrow$, is defined:
\begin{center}
\begin{tabular}{c|c||c}
$x$&$y$&$x\rightarrow y$\\
$\top$&$\top$&$\top$\\
$\top$&$\bot$&$\bot$\\
$\bot$&$\top$&$\top$\\
$\bot$&$\bot$&$\top$
\end{tabular}
\end{center}
\section*{Material biconditional}
The \textit{material biconditional} operation (also called an \textit{if and only if} or \textit{equivalence}), denoted $\leftrightarrow$, is defined:
\begin{center}
\begin{tabular}{c|c||c}
$x$&$y$&$x\leftrightarrow y$\\
$\top$&$\top$&$\top$\\
$\top$&$\bot$&$\bot$\\
$\bot$&$\top$&$\bot$\\
$\bot$&$\bot$&$\top$
\end{tabular}
\end{center}
\section*{Proposition 7}
The expressions $x\rightarrow y$ and $(\lnot x)\lor y$ are logically equivalent.
\begin{center}
\begin{tabular}{c|c|c|c|c}
$x$&$y$&$x\rightarrow y$&$\lnot x$&$(\lnot x)\lor y$\\
$\top$&$\top$&$\top$&$\bot$&$\top$\\
$\top$&$\bot$&$\bot$&$\bot$&$\bot$\\
$\bot$&$\top$&$\top$&$\top$&$\top$\\
$\bot$&$\bot$&$\top$&$\top$&$\top$
\end{tabular}
\end{center}
The columns for $x\rightarrow y$ and $(\lnot x)\lor y$ are the same, and therefore these expressions are logically equivalent.
\end{document}
