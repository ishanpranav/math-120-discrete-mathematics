\documentclass[12pt]{article}
\usepackage[english]{babel}
\usepackage[letterpaper,top=2cm,bottom=2cm,left=3cm,right=3cm,marginparwidth=1.75cm]{geometry}
\usepackage{amsmath}
\usepackage{amsfonts}
\usepackage{graphicx}
\title{MATH-UA 120 Section 21}
\author{Ishan Pranav}
\date{October 16, 2023}
\begin{document}
\maketitle
\section*{Well-ordering principle}
Let $X$ be a set such that $x\neq\emptyset$ and $X\subseteq\mathbb{N}$. Then there exists $x\in X$ where $x$ is the least element of $X$.
\section*{Fibonacci numbers}
The \textit{Fibonacci numbers} are the list of integers $(1,1,2,3,5,8,\dots)=(F_0,F_1,F_2,\dots)$ where
\begin{align*}
F_0&=1,\\
F_1&=1,&\text{ and}\\
F_n&=F_{n-1}+F_{n-2},&\text{ for }n\geq 2.
\end{align*}
\section*{Interesting}
Let $n\in\mathbb{N}$. Then $n$ is \textit{interesting}.

Assume, for the sake of contradiction, that the claim is false. Let $X$ be the set of counterexamples. That is, $X$ is the set of natural numbers that are \textit{not} interesting. Then $X=\{n\in\mathbb{N}\text{ where }n\text{ is not interesting}\}\neq\emptyset$. By the well-ordering principle, since $X\neq\emptyset$ and $X\subseteq\mathbb{N}$, there exists $x\in X$ that is the least element of $X$.

Of course, 0 is an interesting number. For example, it is the identity element for addition. So $x\neq 0$. Similarly, $x\neq 1$ because 1 is the only unit in $\mathbb{N}$. Similarly, $x\neq 2$ because 2 is the only even prime. But $x$ is the first natural number that is not interesting. This makes $x$ interesting, even while $x\in X$---which is absurd. So our assumption is false.

Hence every natural number is interesting. $\square$
\section*{Proposition}
Let $n\in\mathbb{N}$. Then $n$ is either even or odd.
\section*{Proposition}
Let $n\in\mathbb{Z}$ such that $n>0$. The sum of the first $n$ odd natural numbers is $n^2$. That is,

\[\sum_{i=1}^{n}{(2i-1)}=n^2.\]
\section*{Proposition}
Let $n\in\mathbb{N}$. If $a\neq 0$ and $a\neq 1$, then

\[\sum_{i=0}^{n}{a^i}=\frac{a^{n+1}-1}{a-1}.\]
\section*{Proposition}
Let $n\in\mathbb{Z}$ such that $n\geq 5$. Then $2^n>n^2$.
\section*{Proposition}
Let $n\in\mathbb{Z}$. Then $F_n\leq 1.7^n$.
\end{document}