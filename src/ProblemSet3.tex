\documentclass{article}
\usepackage{ifxetex}
\ifxetex
  \usepackage{fontspec}
\else
  \usepackage[T1]{fontenc}
  \usepackage[utf8]{inputenc}
  \usepackage{lmodern}
\fi
\title{Problem Set 3}
\author{
    Ishan Pranav
\\  MATH-UA 120 Discrete Mathematics
}
\date{due October 6, 2023}
\usepackage[headings=runin-fixed-nr]{exsheets}
\makeatletter
    \newcommand{\stepenumdepth}{\advance\@enumdepth\@ne}
\makeatother
\SetupExSheets{
    question/pre-body-hook=\stepenumdepth,
    solution/pre-body-hook=\stepenumdepth,
}
\DeclareInstance{exsheets-heading}{runin-nn-np}{default}{
    runin = true,
    title-post-code = .\space,
    join = {
        main[r,vc]title[l,vc](0pt,0pt);
    }
}
\newif\ifshowsolutions
\showsolutionstrue
\ifshowsolutions
    \SetupExSheets{
        question/pre-hook=\itshape,
        solution/headings=runin-nn-np,
        solution/print=true,
        solution/name=Answer
    }%
    \makeatletter%
    \pretocmd{\@title}{Answers to }%
    \makeatother%
\else
    \SetupExSheets{solution/print=false}
\fi
% Bug workaround: http://tex.stackexchange.com/a/146536/1402
%\newenvironment{exercise}{}{}
\RenewQuSolPair{question}{solution}
%\let\answer\solution
%\let\endanswer\endsolution
\usepackage{manfnt}
\newcommand{\danger}{\marginpar[\hfill\dbend]{\dbend\hfill}}
\usepackage{subcaption}
\newcommand{\Z}{\mathbb{Z}}
\newcommand{\R}{\mathbb{R}}
\newcommand{\N}{\mathbb{N}}
\newcommand{\Q}{\mathbb{Q}}
\usepackage{amsmath, amsthm}
\usepackage{amsfonts}
\usepackage{enumerate}
\usepackage{siunitx}
\DeclareSIUnit\pound{lb}
\usepackage{hyperref}
\newtheorem*{theorem}{Theorem}
\newtheorem*{proposition}{Proposition}
\newtheorem*{claim}{Claim}
\theoremstyle{definition}
\newtheorem*{definition}{Definition}
\begin{document}
\maketitle
These are to be written up and turned in to Gradescope.\\
\ifshowsolutions
    \SetupExSheets{solution/print=true}
\else
    \danger
 \underline{ \LaTeX  Instructions:}  You can view the source (\texttt{.tex}) file to get some more examples of \LaTeX{} code.  I have commented the source file in places where new \LaTeX{} constructions are used.
  
  Remember to change \verb|\showsolutionsfalse| to \verb|\showsolutionstrue|
    in the document's preamble 
    (between \verb|\documentclass{article}| and \verb|\begin{document}|)
\fi
\section*{Assigned Problems}
\begin{question}
   \begin{enumerate}
   \item Consider the following subsets of $\N$.
       \begin{align*}
           A &= \text{The set of all even numbers.}\\
           B &= \text{The set of all prime numbers.}\\
           C &= \text{The set of all perfect squares.}\\
           D &= \text{The set of all multiples of 10.}
       \end{align*}
       Using \textbf{only} the symbols $3, A, B, C, D, \N, \in, \subseteq, =, \neq, \cap, \cup, \times, -, \emptyset$, ``('', and  ``)'', rewrite the following statements in set notation. 
           \begin{enumerate}
               \item None of the perfect squares are prime numbers. 
               \item The number 3 is a prime number that is not even.
               \item If you take all the prime numbers, all the even numbers, all the perfect squares, and all the multiples of 10, you still won't have all the natural numbers.
           \end{enumerate}
   \item Consider the following subsets of the set of all students at some university. 
       \begin{align*}
           F &= \text{The set of all freshmen.}\\
           S &= \text{The set of all seniors.}\\
           M &= \text{The set of all math majors.}\\
           C &= \text{The set of all CS majors.}
       \end{align*}
           \begin{enumerate}
               \item Using only the symbols $F, S, M, C, | ~ |, \cap, \cup, -$, and $>$, translate the following statement into the language of set theory. 
               \begin{quote}
                   ``There are more freshmen who aren't math majors than there are senior CS majors.''
               \end{quote}
               \item Translate the following statement in set theory into everyday English. 
               $$(F\cap M)\subseteq C$$
           \end{enumerate}
   \end{enumerate}
\end{question}
\begin{solution}
\begin{enumerate}
    \item\begin{enumerate}
        \item$C\cap B=\emptyset$
        \item$3\in(B-A)$
        \item$N-(B\cup A\cup C\cup D)\neq\emptyset$
    \end{enumerate}
    \item\begin{enumerate}
        \item$|F-M|>|S\cap C|$
        \item Every Freshman pursuing a math degree is also pursuing a CS degree.
    \end{enumerate}
\end{enumerate}
\end{solution}
\begin{question}
Describe explicitly in English the following sets, then give their cardinality.
\begin{enumerate}
\item $\{x \in 2^{\Z} : 5 \in x \}$
\item $\{x \in 2^{\Z} : x \subseteq \{ 1, 2, 3\} \}$
\item $\{x \in 2^{\Z} : x \subseteq \{ 1, 2, \{3, 4\} \} \}$
\item $\{x \in 2^{\Z} : x \in \{ 1, 2, \{3, 4\} \} \}$
\item $\{x \in 2^{\Z} : y \in x \implies y = 0 \}$
\end{enumerate}
\end{question}
\begin{solution}
\begin{enumerate}
    \item The set of all subsets of the integers containing the integer 5. The set is infinite: Its cardinality is infinite.
    \item The set of all subsets of the integers that are subsets of the set containing 1, 2, and 3 only; in other words, the power set of the set containing 1, 2, and 3 only. Its cardinality is $\left|2^{\{1,2,3\}}\right|=2^{|\{1,2,3\}|}=2^3=8$.
    \item The set of all subsets of the integers that are subsets of the set containing the following: 1, 2, and the set containing 3 and 4. That is, $\{\emptyset,\{1\},\{2\},\{1,2\}\}$. The cardinality is 4.
    \item The set of all subsets of the integers that are elements of the set containing the following: 1, 2, and the set containing 3 and 4. That is, $\{{3,4}\}$. Its cardinality is 1.
    \item The set of all subsets of the integers such that every element of every subset is equal to zero. The set can be expressed as $\left\{\emptyset,\{0\}\right\}$ because the condition requires every element in every subset to be 0 (vacuously true for the empty set). Its cardinality is 2.
\end{enumerate}
\end{solution}
\begin{question}
\begin{enumerate}
    \item For each of the following statements, describe it in English, and say if it is true or false (without proof). Then write its negation using quantifiers, and express this negation in English. For instance, the statement $\forall x \in \Z \; x < 0$ means every integer is negative, and it is false. Its negation is $\exists x \in \Z \; x \geq 0$, which means that there exists a nonnegative integer.
	\begin{enumerate}
		\item $\forall x \in \Z \; \exists y \in \Z \; x^2 + y = 4$
		\item $\exists y \in \Z \; \forall x \in \Z \; x^2 + y = 4$
		\item $\forall n \in \Z \; \exists k \in \Z \; \exists d \in \Z \; k+ n = 2d$
		\item $\exists n \in \Z \; \forall k \in \Z \; \exists d \in \Z \; k+ n = 2d$
	\end{enumerate}
	\item For each of the statements (iii) and (iv): prove it if it is true, or prove the negation if it is false. These proofs are short.
\end{enumerate}
\end{question}
\begin{solution}
\begin{enumerate}
    \item\begin{enumerate}
        \item The statement $\forall x\in\Z\;\exists y\in\Z\;x^2+y=4$ means that every perfect square can be added to some integer to obtain a sum of 4. The statement is true.
        
        The negation of this statement is $\exists x\in\Z\;\forall y\in\Z\;x^2+y\neq 4$. The negation means that there is a perfect square that can be added to any integer to obtain a sum other than 4.
        \item The statement $\exists y\in\Z\;\forall x\in\Z\;x^2+y=4$ means that there is an integer that can be added to any perfect square to obtain a sum of 4. The statement is false.

        The negation of this statement is $\forall y\in\Z\;\exists x\in\Z\;x^2+y\neq 4$. The negation means that every perfect square can be added to some integer to obtain a sum other than 4.
        \item The statement $\forall n\in\Z\;\exists k\in\Z\;\exists d\in\Z\;k+n=2d$ means that any integer can be added to some integer to obtain an even sum. The statement is true.

        The negation of this statement is $\exists n\in\Z\;\forall k\in\Z\;\forall d\in\Z;k+n\neq 2d$. The negation means that there is some integer that can be added to any integer to obtain a sum that is not even.
        \item The statement $\exists n\in\Z\;\forall k\in\Z\;\exists d\in\Z\;k+n=2d$ means that there is an integer that can be added to any integer to obtain an even sum. The statement is false.

        The negation of this statement is $\forall n\in\Z\;\exists k\in\Z\;\forall d\in\Z;k+n\neq 2d$. The negation means that any integer can be added to some integer to obtain a sum that is not even.
    \end{enumerate}
    \item\begin{enumerate}
        \item[iii.] The statement is true, and we will prove it.
        \begin{claim}
        $\forall n\in\Z\;\exists k\in\Z\;\exists d\in\Z\;k+n=2d.$
        \end{claim}
        \begin{proof}
        Let $n\in\Z$.
        
        Suppose $n$ is even. Then $2\mid n$, so $n=2d$ for some $d\in\Z$. Consider $k=0\in\Z$. Then $k+n=0+2d=2d$.

        Suppose instead $n$ is odd. Then $n=(2d+1)$ for some $d\in\Z$. Consider $k=(-1)\in\Z$. Then $k+n=(-1)+(2d+1)=2d$.

        Hence, for all $n\in\Z$, there exists $k,d\in\Z$ such that $k+n=2d$.
        \end{proof}
        \item[iv.] The statement is false, but we will prove its negation.
        \begin{claim}
        $\forall n\in\Z\;\exists k\in\Z\;\forall d\in\Z\;k+n\neq 2d.$
        \end{claim}
        \begin{proof}
        Let $n\in\Z$.
        
        Suppose $n$ is even. Then $2\mid n$ so $n=2a$ for some $a\in\Z$. Consider $k=1\in\Z$. Then $k+n=1+2a$. Then $k+n$ is odd. But $2d$ is even. Therefore $k+n\neq 2d$.
        
        Suppose instead $n$ is odd. Then $n=2a+1$ for some $a\in\Z$. Consider $k=0\in\Z$. Then $k+n=1+2a$. Then $k+n$ is odd. But $2d$ is even. Therefore $k+n\neq 2d$.
        
        Hence, for all $n\in\Z$, there exists $k\in\Z$ such that for all $d\in\Z, k+n\neq 2d$.
        \end{proof}
    \end{enumerate}
\end{enumerate}
\end{solution}
\begin{question}
   Let $I=\{1, 2, \dots, n\}$. Given a collection of sets $\{A_1,A_2,\dots, A_n\}$, denoted by $\{A_i\}_{i\in I}$. $\{A_i\}_{i\in I}$ is said to be \textbf{disjoint} if $\cap_{i\in I}A_i=\emptyset$, and it is said to be \textbf{pairwise disjoint} if $A_i\cap A_j=\emptyset$ whenever $i\neq j$. What is the difference between a \textbf{disjoint} collection of sets and a \textbf{pairwise disjoint} collection of sets?
\end{question}
\begin{solution}
A \textit{disjoint} collection of sets is one such that the intersection of all the sets is equal to the empty set. In other words, there is no element such that every set contains that element. Meanwhile, a \textit{pairwise disjoint} collection of sets is one such that the intersection of every pair of sets is equal to the empty set. In other words, no two sets in a pairwise disjoint collection have elements in common. Pairwise disjoint implies disjoint, but disjoint does not necessarily imply pairwise disjoint. The difference becomes apparent when there are more than two sets in a collection. For example, the collection of sets $\{\{0,1\},\{1,2\},\{0,2\}\}$ is disjoint because there is no element that all three sets have in common ($\bigcap_{i\in I}{A_i}=\emptyset$), but it is not pairwise disjoint because $\{0,1\}\cap\{1,2\}=\{1\}$ ($A_i\cap A_j\neq\emptyset$).
\end{solution}
\begin{question}
   Let $E$ be the set of all even integers and let $O$ be the set of all odd integers. Let $X=\{ n\in \Z : n = x+y \text{ for some } x, y\in O\}$. Prove $X=E$.
\end{question}
\begin{solution}
\begin{claim}
Let $E=\{x\in\Z:2\mid x\}$ and $O=\{x\in\Z:x=2k+1\text{ for some }k\in\Z\}$. Then $X=E$.
\end{claim}
\begin{proof}
$\subseteq$) Let $n\in X$. Then there exists $x\in O$ and $y\in O$ such that $n=x+y$. Since $x\in O$, there exists $a\in\Z$ such that $x=2a+1$. Since $y\in O$, there exists $b\in\Z$ such that $y=2b+1$. Observe
\begin{align*}
n
&=x+y\\
&=(2a+1)+(2b+1)\\
&=2a+2b+2\\
&=2(a+b+1).
\end{align*}
Since $a+b+1\in\Z$ and $n=2(a+b+1)$, $2\mid n$. Therefore, $n$ is even, so $n\in E$.
\newline

$\supseteq$) Let $n\in E$. Then there exists $a\in\Z$ such that $n=2a$. Let $p\in\Z$ and $q\in\Z$ such that $a=p+q+1$. Observe
\begin{align*}
n
&=2a\\
&=2(p+q+1)\\
&=2p+2q+2\\
&=2p+1+2q+1\\
&=(2p+1)+(2q+1).
\end{align*}
Let $x=2p+1\in\Z$ and $y=2q+1\in\Z$. Since $p$ and $q$ are integers, $x$ is odd and $y$ is odd, so $x\in O$ $y\in O$. Therefore $n=x+y$  for some $x,y\in O$.
\newline

Hence, $X=E$.
\end{proof}
\end{solution}
\end{document}