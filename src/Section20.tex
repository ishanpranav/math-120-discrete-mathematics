\documentclass[12pt]{article}
\usepackage[english]{babel}
\usepackage[letterpaper,top=2cm,bottom=2cm,left=3cm,right=3cm,marginparwidth=1.75cm]{geometry}
\usepackage{amsmath}
\usepackage{amsfonts}
\usepackage{graphicx}
\title{MATH-UA 120 Section 20}
\author{Ishan Pranav}
\date{October 11, 2023}
\begin{document}
\maketitle
\section*{Proposition}
No integer is both even and odd.
\section*{Proposition}
Let $a$ and $b$ be numbers with $a\neq 0$. There is at most one number $x$ with $ax+b=0$.

Assume, for the sake of contradiction, $ax+b=0$, and $ay+b=0$, but $x\neq y$. This gives $ax+b=ay+b$. Thus $ax=ay$. Since $a\neq 0$, $x=y$, even while $x\neq y$---which is absurd. So our assumption is false.

Hence, for all numbers $a$ and $b$ where $a\neq 0$, there is at most one number $x$ with $ax+b=0.~\square$
\end{document}