\documentclass[12pt]{article}
\usepackage[english]{babel}
\usepackage[letterpaper,top=2cm,bottom=2cm,left=3cm,right=3cm,marginparwidth=1.75cm]{geometry}
\usepackage{amsmath}
\usepackage{amsfonts}
\usepackage{graphicx}
\DeclareMathOperator{\dom}{dom}
\DeclareMathOperator{\im}{im}
\title{MATH-UA 120 Section 25}
\author{Ishan Pranav}
\date{November 1, 2023}
\begin{document}
\maketitle
\section*{Pigeonhole principle}
Let $A$ and $B$ be finite sets and let $f:A\to B$. If $|A|>|B|$, then $f$ is not injective. If $|A|<|B|$, then $f$ is not surjective.
\section*{Lattice point}
A point whose coordinates are both integers is called a \textit{lattice point}.
\section*{Sequence}
A \textit{sequence} is simply a list.
\section*{Subsequence}
Given a sequence, a \textit{subsequence} is a list formed by deleting elements from the original list and keeping the remaining elements in the same order in which they originally appeared.
\section*{Decreasing subsequence}
If the elements of a subsequence are in decreasing order, then we call the subsequence a \textit{decreasing subsequence}.
\section*{Increasing subsequence}
If the elements of a subsequence are in increasing order, then we call the subsequence an \textit{increasing subsequence}.
\section*{Monotone sequence}
A sequence that is either increasing or decreasing is called \textit{monotone}.
\section*{Erdős--Szekeres theorem}
Let $n\in\mathbb{Z}$ such that $n>0$. Every sequence of $n^2+1$ distinct integers must contain a monotone subsequence of length $n+1$.
\section*{Cantor's theorem}
\noindent\textit{Claim. }Let $A$ be a set. If $f:A\to 2^A$, then $f$ is not surjective.\newline 

\noindent\textit{Proof. } Let $A$ be a set and let $f:A\to 2^A$. Let $B=\{x\in A:x\notin f(x)\}$.\newline

\noindent Assume, for the sake of contradiction, that there exists $a\in A$ such that $f(a)=B$. Then $a\in B$ or $a\notin B$.\newline

\noindent Suppose $a\in B$. Since $B=f(a)$, we have $a\in f(a)$, even while $a\in B$---which is absurd. So our assumption is false.\newline

\noindent Suppose $a\notin B$. Since $B=f(a)$, we have $a\notin f(a)$, even while $a\notin B$---which is absurd. So our assumption is false.\newline

\noindent In all cases, our assumption is false: There does not exist $a\in A$ such that $f(a)=B$. There exists $B\in 2^A$ for which there does not exist $a\in A$ such that $f(a)=B$. Hence, $f$ is not surjective. $\square$
\section*{Proposition}
Let $n\in\mathbb{N}$. Then there exist $a,b\in\mathbb{Z}$ such that $a>0$ and $b>0$ and such that $a\neq b$ and such that $10\mid\left(n^a-n^b\right)$.
\section*{Proposition}
Given five distinct lattice points in the plane, at least one of the line segments determined by these points has a lattice point as its midpoint.
\end{document}