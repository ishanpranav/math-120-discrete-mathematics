\documentclass[12pt]{article}
\usepackage[english]{babel}
\usepackage[letterpaper,top=2cm,bottom=2cm,left=3cm,right=3cm,marginparwidth=1.75cm]{geometry}
\usepackage{amsfonts}
\usepackage{amsmath}
\usepackage{amssymb}
\usepackage{graphicx}
\renewcommand{\labelenumi}{\alph{enumi}.}
\title{MATH-UA 120 Section 7}
\author{Ishan Pranav}
\date{September 12, 2023}
\begin{document}
\maketitle
\section*{And}
The operation \textit{and}, denoted $\land$, is defined:

\[\top\land\top=\top,\]

\[\top\land\bot=\bot,\]

\[\bot\land\top=\bot,\]

\[\bot\land\bot=\bot.\]
\section*{Or}
The operation \textit{or}, denoted $\lor$, is defined:

\[\top\lor\top=\top,\]

\[\top\lor\bot=\top,\]

\[\bot\lor\top=\top,\]

\[\bot\lor\bot=\bot.\]
\section*{Not}
The operation \textit{not}, denoted $\lnot$, is defined:

\[\lnot\top=\bot,\]

\[\lnot\bot=\top.\]
\section*{Proposition 6}
The Boolean expressions $\lnot(x\land y)$ and $(\lnot x)\lor(\lnot y)$ are logically equivalent.

\begin{center}
\begin{tabular}{c|c||c|c||c|c|c}
$x$&$y$&$x\land y$&$\lnot(x\land y)$&$\lnot x$&$\lnot y$&$(\lnot x)\lor(\lnot y)$\\
$\top$&$\top$&$\top$&$\bot$&$\bot$&$\bot$&$\bot$\\
$\top$&$\bot$&$\bot$&$\top$&$\bot$&$\top$&$\top$\\
$\bot$&$\top$&$\bot$&$\top$&$\top$&$\bot$&$\top$\\
$\bot$&$\top$&$\bot$&$\top$&$\top$&$\bot$&$\top$
\end{tabular}
\end{center}
\section*{Commutative property of \textit{and}}
\[x\land y=y\land x.\]
\section*{Commutative property of \textit{or}}
\[x\lor y=y\lor x.\]
\section*{Associative property of \textit{and}}
\[(x\land y)\land z=x\land(y\land z).\]
\section*{Associative property of \textit{or}}
\[(x\lor y)\lor z=x\lor(y\lor z).\]
\section*{\textit{True} identity}
\[x\land\top=x.\]
\section*{\textit{False} identity}
\[x\lor\bot=x.\]
\section*{Idemopotency}
\[x\land x=x.\]
\[x\lor x=x.\]
\section*{Double negative}
\[\lnot(\lnot x)=x.\]
\section*{Distributive property of \textit{and}}
\[x\land(y\lor z)=(x\land y)\lor(x\land z).\]
\section*{Distributive property of \textit{or}}
\[x\lor(y\land z)=(x\lor y)\land(x\lor z).\]
\section*{Inverse element}
\[x\land(\lnot x)=\bot.\]
\[x\lor(\lnot x)=\top.\]
\section*{De Morgan's laws}
\[\lnot(x\land y)=(\lnot x)\lor(\lnot y),\]

\[\lnot(x\lor y)=(\lnot x)\land(\lnot y).\]
\section*{Implication}
The \textit{material conditional} operation (also called an \textit{if-then} or \textit{implication}), denoted $\rightarrow$, is defined:
\begin{center}
\begin{tabular}{c|c||c}
$x$&$y$&$x\rightarrow y$\\
$\top$&$\top$&$\top$\\
$\top$&$\bot$&$\bot$\\
$\bot$&$\top$&$\top$\\
$\bot$&$\bot$&$\top$
\end{tabular}
\end{center}
\section*{Equivalence}
The \textit{material biconditional} operation (also called an \textit{if and only if} or \textit{equivalence}), denoted $\leftrightarrow$, is defined:
\begin{center}
\begin{tabular}{c|c||c}
$x$&$y$&$x\leftrightarrow y$\\
$\top$&$\top$&$\top$\\
$\top$&$\bot$&$\bot$\\
$\bot$&$\top$&$\bot$\\
$\bot$&$\bot$&$\top$
\end{tabular}
\end{center}
\section*{Proposition 7}
The expressions $x\rightarrow y$ and $(\lnot x)\lor y$ are logically equivalent.
\begin{center}
\begin{tabular}{c|c|c|c|c}
$x$&$y$&$x\rightarrow y$&$\lnot x$&$(\lnot x)\lor y$\\
$\top$&$\top$&$\top$&$\bot$&$\top$\\
$\top$&$\bot$&$\bot$&$\bot$&$\bot$\\
$\bot$&$\top$&$\top$&$\top$&$\top$\\
$\bot$&$\bot$&$\top$&$\top$&$\top$
\end{tabular}
\end{center}
The columns for $x\rightarrow y$ and $(\lnot x)\lor y$ are the same, and therefore these expressions are logically equivalent.
\section*{Tautology}
A Boolean expression is a \textit{tautology} if the evaluation of all possible values of its variables is true.
\section*{Contradiction}
A Boolean expression is a \textit{contradiction} if the evaluation of all possible values of its variables is false.
\section*{Contingency}
A Boolean expression is a \textit{contingency} if its evaluation is sometimes true and sometimes false.
\section{Calculations}
\begin{enumerate}
    \item$\top\land\top\land\top\land\top\land\bot$=$\bot$.
    \item$(\lnot\top)\lor\top=\top$.
    \item$\lnot(\top\lor\top)=\bot$.
    \item$(\top\lor\top)\land\bot=\bot$.
    \item$\top\lor(\top\land\bot)=\top$.
\end{enumerate}
\section{Prove: $(x\land y)\lor(x\land\lnot y)=x$}
\begin{align*}
(x\land y)\lor(x\land\lnot y)
&=x\land(y\lor\lnot y),&\textit{distributive property;}\\
&=x\land\top,&\textit{inverse elements;}\\
&=x,&\textit{identity.}&\,\blacksquare
\end{align*}
\section{Prove: $x\rightarrow y=(\lnot y)\rightarrow(\lnot x)$}
\begin{align*}
x\rightarrow y
&=(\lnot x)\lor y,&\textit{Proposition 7;}\\
&=y\lor(\lnot x),&\textit{commutative property;}\\
&=\lnot(\lnot y)\lor(\lnot x),&\textit{double negative;}\\
&=(\lnot y)\rightarrow(\lnot x),&\textit{Proposition 7.}
\end{align*}
We conclude that an \textit{if-then} statement ($x\rightarrow y$) is logically equivalent to its contrapositive, $(\lnot y)\rightarrow(\lnot x)$. $\blacksquare$
\section{Prove: $x\leftrightarrow y=(x\rightarrow y)\land(y\rightarrow x)$}
\begin{align*}
x\leftrightarrow y
&=((\lnot x)\lor y)\land((\lnot y)\lor x)\\
&=((\lnot x)\lor y)\land(y\rightarrow x)\\
&=(x\rightarrow y)\land(y\rightarrow x).~\blacksquare
\end{align*}
\section{Prove: $x\rightarrow y=(\lnot x)\lor y$}
\begin{center}\begin{tabular}{c|c||c||c|c}
$x$&$y$&$x\rightarrow y$&$\lnot x$&$(\lnot x)\lor y$\\
1&1&1&0&1\\
1&0&0&0&0\\
0&1&1&1&1\\
0&0&1&1&1
\end{tabular}\end{center}
We conclude that $x\rightarrow y$ implies $(\lnot x)\lor y$. $\blacksquare$
\end{document}
