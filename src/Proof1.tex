\documentclass[9pt]{article}
\usepackage[margin=1in]{geometry}
\usepackage[utf8]{inputenc}
\usepackage{amsmath}
\usepackage{amssymb}
\usepackage{amsthm}
\title{Polished Proof 1}
\author{\textbf{Ishan Pranav}}
\begin{document}
\maketitle
\section*{Claim}
Let $a\in \mathbb{Z}$. $14\mid a$ if and only if $7\mid a$ and $2\mid a$.
\section*{Proof}
$\Rightarrow$) First, we will demonstrate that if $14\mid a$, then $7\mid a$ and $2\mid a$. Suppose $14\mid a$. Since $14\mid a$, there exists $n\in\mathbb{Z}$ such that $a=14n$. Note $a=7(2n)$ by multiplication. There exists $(2n)\in\mathbb{Z}$ such that $a=7(2n)$. Thus, $7\mid a$. Note also $a=2(7n)$ by multiplication. There exists $(7n)\in\mathbb{Z}$ such that $a=2(7n)$. Thus, $2\mid a$. Therefore, $7\mid a$ and $2\mid a$.
\newline

\noindent$\Leftarrow$) Next, we will demonstrate that if $7\mid a$ and $2\mid a$, then $14\mid a$. Suppose $7\mid a$ and $2\mid a$. Since $7\mid a$, there exists $b\in\mathbb{Z}$ such that $a=7b$. We know that for all integers, the terms ``even'' and ``odd'' are mutually exclusive. Since $b\in\mathbb{Z}$, $b$ is either even or odd.

Suppose $b$ is odd. Then there exists $c\in\mathbb{Z}$ such that $b=2c+1$. Observe
\begin{align*}
b&=2c+1&\textit{definition of odd}\\
7b&=7(2c+1)&\textit{multiplication property of equality}\\
7b&=14c+7&\textit{distributive property of multiplication}\\
7b&=14c+6+1&\textit{addition}\\
7b&=2(7c+3)+1.&\textit{distributive property of multiplication}
\end{align*}

There exists $(7c+3)\in\mathbb{Z}$ such that $7b=2(7c+3)+1$. Thus, $7b$ is odd. Note $2\mid a$, so $a$ is even. However, $a=7b$, and $7b$ is odd, so $a$ is odd. This is absurd. The statements ``$a$ is even'' and ``$a$ is odd'' lead to a contradiction, and hence our supposition that $b$ is odd is false: $b$ is not odd.

Since $b$ is not odd, $b$ is even. Since $b$ is even, 2 is a factor of $b$. Since $2\mid b$, there exists $m\in\mathbb{Z}$ such that $b=2m$. Observe
\begin{align*}
b&=2m&\textit{definition of even}\\
7b&=7(2m)&\textit{multiplication property of equality}\\
a&=7(2m)&\textit{substitution property of equality}\\
a&=14m.&\textit{multiplication}
\end{align*}

There exists $m\in\mathbb{Z}$ such that $a=14m$. Therefore, $14\mid a$.
\newline

\noindent We conclude that $14\mid a$ if and only if $7\mid a$ and $2\mid a$.\,\,\blacksquare$
\section*{Reflection}
This proof is built upon concepts from Sections 1-8 of the textbook. I did not find it daunting or impossible. From the phrase ``A if and only if B,'' I immediately knew that the demonstration would involve two parts, one proving ``if B, then A'' and the other proving ``A only if B.'' So while I wasn't working off of a rigid ``template,'' I was scaffolding a framework and filling in the gaps as I went.

I began the proof by converting the problem into a claim, establishing the key variable ($a$) and its type (integer). Then, I set up the sub-claims and conclusions for each part and began working backward. The first part (if $14\mid a$, then $7\mid a$ and $2\mid a$) was straightforward. I immediately knew how to approach it and completed it.

The second part required additional cases and some tricky logic. My first approach was to make a claim about $a\cdot a$, attempting to use the fact that 7 and 2 were factors of $a$, and the fact that $7\cdot 2=14$ to my advantage. That meant I would have to divide by $a$ to get the equation into the form $a=(\dots)$. For the first half hour, every algebraic approach I tried involved some form of division that destroyed any guarantee that my result would be an integer.  

After taking a break and revisiting the problem after dinner, I realized that I could circumvent the division issue if I could guarantee that my expression was even. If it were, I could divide by 2 and still have an integer! This was the ``\textit{ah-ha}'' moment. I realized there were two cases, one where $b$ is even, and the other where $b$ is odd---which (fortunately) was absurd. With $b$ guaranteed to be even, my roadblocks had been eliminated. Within the next 15 minutes, I finished the proof. Overall, the process took about 2 hours. My ``startup time'' is slow since I do all my thinking directly in the \LaTeX\, editor. Still, I prefer typing rather than paper, since once I've solved it, there's nothing more to be done.
\end{document}
