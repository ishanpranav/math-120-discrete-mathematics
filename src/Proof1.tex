\documentclass[9pt]{article}
\usepackage[margin=1in]{geometry}
\usepackage[utf8]{inputenc}
\usepackage{amsmath}
\usepackage{amssymb}
\usepackage{amsthm}
\title{Polished Proof 1}
\author{\textbf{Ishan Pranav}}
\begin{document}
\maketitle
\noindent\textit{Claim. }Let $a\in \mathbb{Z}$. $14\mid a$ if and only if $7\mid a$ and $2\mid a$.
\newline

\noindent\textit{Proof. }Let $a\in\mathbb{Z}$. We will demonstrate that $14\mid a$ if and only if $7\mid a$ and $2\mid a$.
\newline

\noindent$\Rightarrow$) Suppose $14\mid a$. Then there exists $n\in\mathbb{Z}$ such that $a=14n$. Note $a=7(2n)$. There exists $(2n)\in\mathbb{Z}$ such that $a=7(2n)$. Thus, $7\mid a$. Note also $a=2(7n)$. There exists $(7n)\in\mathbb{Z}$ such that $a=2(7n)$. Thus, $2\mid a$. Therefore, $7\mid a$ and $2\mid a$. 
\newline

\noindent$\Leftarrow$) Suppose $7\mid a$ and $2\mid a$. Then there exists $b\in\mathbb{Z}$ such that $a=7b$. Since $2\mid a$, there exists $c\in\mathbb{Z}$ such that $a=2c$. Thus $a=7b=2c$. There exists $c\in\mathbb{Z}$ such that $7b=2c$, so $2\mid 7b$. Thus $7b$ is even, so either 7 is even or $b$ is even; but 7 is not even, so $b$ is even. Since $b$ is even, $2\mid b$ and there exists $d\in\mathbb{Z}$ such that $b=2d$. Observe
\begin{align*}
a&=7b\\
a&=7(2d)\\
a&=14d.
\end{align*}
There exists $d\in\mathbb{Z}$ such that $a=14d$. Therefore, $14\mid a$.
\newline

\noindent Hence, $14\mid a$ if and only if $7\mid a$ and $2\mid a$.$\,\,\blacksquare$
\section*{Reflection}
After the first draft, I learned about proof by contradiction and realized that the logic that I used was effectively a proof by contradiction---which wasn't a technique that we used in class. This meant I would have to rework the proof to conclude that the integer $b$ mentioned in the proof is even, this time using a direct proof technique. My new approach was to use the fact that $2a=7b$, and thus $7b$ is even. Then, I employed one of the fundamental properties of even and odd that we had covered in class, the idea that if there are two integer factors of an even number, one of the factors must be even. I incorporated this strategy, and I also cleaned up some of the proof to make it more concise and straightforward, removing unnecessary lines of explanation wherever I saw them. Compared with my first draft, the bulk of the logic remains the same as in the final draft. However, I believe the new proof is more elegant, straightforward, and streamlined. There are fewer redundant explanations, and the entire argument follows from definitions from start to finish.
\end{document}
