\documentclass{article}
\usepackage{ifxetex}
\ifxetex
  \usepackage{fontspec}
\else
  \usepackage[T1]{fontenc}
  \usepackage[utf8]{inputenc}
  \usepackage{lmodern}
\fi
\title{Problem Set 6}
\author{%
    Ishan Pranav
\\  MATH-UA 120 Discrete Mathematics
}
\date{due November 10, 2023}
\usepackage[headings=runin-fixed-nr]{exsheets}
\makeatletter
    \newcommand{\stepenumdepth}{\advance\@enumdepth\@ne}
\makeatother
\SetupExSheets{
    question/pre-body-hook=\stepenumdepth,
    solution/pre-body-hook=\stepenumdepth,
}
\DeclareInstance{exsheets-heading}{runin-nn-np}{default}{
    runin = true,
    title-post-code = .\space,
    join = {
        main[r,vc]title[l,vc](0pt,0pt);
    }
}
\newif\ifshowsolutions
\showsolutionstrue
\ifshowsolutions
    \SetupExSheets{
        question/pre-hook=\itshape,
        solution/headings=runin-nn-np,
        solution/print=true,
        solution/name=Answer
    }%
    \makeatletter%
    \pretocmd{\@title}{Answers to }%
    \makeatother%
\else
    \SetupExSheets{solution/print=false}
\fi

% Bug workaround: http://tex.stackexchange.com/a/146536/1402
%\newenvironment{exercise}{}{}
\RenewQuSolPair{question}{solution}
%\let\answer\solution
%\let\endanswer\endsolution
\usepackage{manfnt}
\newcommand{\danger}{\marginpar[\hfill\dbend]{\dbend\hfill}}
\newcommand{\Z}{\mathbb{Z}}
\newcommand{\N}{\mathbb{N}}
\newcommand{\R}{\mathbb{R}}
\newcommand{\id}{\operatorname{id}}
\usepackage{tikz}
\usepackage{amsmath, amsthm}
\usepackage{amsfonts}
\usepackage{siunitx}
\DeclareSIUnit\pound{lb}
\DeclareMathOperator{\dom}{dom}
\DeclareMathOperator{\im}{im}
\usepackage{hyperref}
\newtheorem*{theorem}{Theorem}
\theoremstyle{definition}
\newtheorem*{definition}{Definition}
\begin{document}
\maketitle
These are to be written up in \LaTeX{} and turned in to Gradescope.\\
\ifshowsolutions
    \SetupExSheets{solution/print=true}
\else
    \danger
 \underline{ \LaTeX{}  Instructions:}  You can view the source (\texttt{.tex}) file to get some more examples of \LaTeX{} code.  I have commented the source file in places where new \LaTeX{} constructions are used.
  
  Remember to change \verb|\showsolutionsfalse| to \verb|\showsolutionstrue|
    in the document's preamble 
    (between \verb|\documentclass{article}| and \verb|\begin{document}|)
\fi
\section*{Assigned Problems}
\begin{question}
    Suppose that we have two piles of cards each containing $n$ cards. Two players play a game as follows. Each player, in turn, chooses one pile and then removes any number of cards, but at least one, from the chosen pile. The player who removes the last card wins the game. Show that the second player can always win the game. 
    \textit{Hint: Use strong induction.}
\end{question}
\begin{solution}
~\newline

\noindent\textit{Claim. }Let there be two piles each containing $n$ cards. Two players play a game as follows. Each player, in turn, chooses one pile and then removes at least one card from it. The player who removes the last card wins. The second player can always win.\newline

\noindent\textit{Proof. }Let there be two piles. At the \textit{beginning} of a given turn $j\in\mathbb{N}$, let $a_j$ be the number of cards in one pile and $b_j$ be the number of cards in the other pile. Suppose there is a first player and a second player, such that the first player takes the first turn and the second player takes the second turn. The first turn has $j=1$. Let $n\in\mathbb{N}$ be the initial number of cards in each pile, so $a_1=b_1=n$. For every turn $j$ where $j$ is odd, the first player chooses one pile and then removes at least one card, but no more than the number of cards in the pile. For every turn $j$ where $j$ is not odd, the second player chooses one pile and then removes at least one card, but no more than the number of cards in the pile. If, at the beginning of turn $j+1$, we have $a_{j+1}=0$ and $b_{j+1}=0$, the game ends. If $j+1$ is odd, then the second player wins; otherwise, the first player wins.\newline

\noindent First, we will eliminate the possibility that $n=0$. Assume, for the sake of contradiction, that $n=0$. We know
\[a_1=b_1=n=0.\]
On turn $j=1$, the first player chooses one pile and then removes $x_0\in\mathbb{N}$ cards from it, where $1\leq x_0\leq n$. But $n=0$, so $1\leq 0$, which is a contradiction. Thus our assumption is false: $n\neq 0$.\newline

\noindent Now we will demonstrate that for every initial number of cards $n>0$, the second player can always win by induction on $n$.

\begin{description}
\item[Basis case.] Consider $n=1$. Then $a_1=b_1=1$. On turn $j=1$, suppose the first player chooses one pile and removes one card. Then $a_2=0$ and $b_2=1$. On turn $j=2$, if the second player chooses the other pile, then $a_3=0$ and $b_3=0$. Thus the second player wins. Therefore, if $n=1$, then the second player can win.
\item[Inductive hypothesis.] Let $k\in\mathbb{N}$ such that $k\geq 1$. Assume that if $n=1,2,3,\dots,\text{ or }k$, then the second player can always win. In other words, if for some natural number $k$ greater than 1, the number of cards in both piles the same natural number $n$ between 1 and $k$, then the second player can win.
\item[Inductive step.] Consider $n=k+1$.

We know
\[a_1=b_1=n=(k+1).\]
On turn $j=1$, the first player chooses $x\in\mathbb{N}$ cards where $1\leq x\leq (k+1)$. So $a_2=(k+1)-x$ and $b_2=(k+1)$. Then on turn $j=2$, the second player can choose the same $x$ cards from the other pile, so $a_3=(k+1)-x$ and $b_3=(k+1)-x$. Since $1\leq x\leq(k+1)$, we have $0\leq (a_3=b_3)\leq k$.

Suppose $a_3=b_3=0$. Then the second player wins.

Suppose instead $(a_3=b_3)\neq 0$. Then $1\leq(a_3=b_3)\leq k$. Therefore the second player can win by the inductive hypothesis, thus completing the inductive step.
\end{description}
Hence, by the principle of mathematical induction, the second player can always win.$~\square$
\end{solution}
\begin{question}
    For each of the following functions, say if it is one-to-one and / or onto? Prove or disprove each statement.
    \begin{enumerate}
	\item $f : \Z \to \Z$ with $f(n) = n^2 + 1$ for $n \in \Z$.
	\item $f : \Z \to \Z$ with $f(n) = n/2$ if $n$ is even, and $f(n) = 0$ if $n$ is odd.
	\item $f : \R \to \R$ with $f(x) = 1/x$ if $x \neq 0$, and $f(0) = 0$.
	\item $f: \N \to \N$ with $f(n) = 2^n$ if $n$ is even and $f(n) = n$ if $n$ is odd.
	\item $f : \mathcal{P}(\Z) \to \mathcal{P}(\Z)$ with $f(A) = A \cup \{ 0 \}$ for $A \in \mathcal{P}(\Z)$.
    \end{enumerate}
\end{question}
\begin{solution}
\begin{enumerate}
\item\textit{Claim. }Let $f:\Z\to\Z$ with $f(n)=n^2+1$ for $n\in\Z$. Then $f$ is not one-to-one and $f$ is not onto.\newline

\textit{Proof. }Let $f:\Z\to\Z$ with $f(n)=n^2+1$ for $n\in\Z$. We want to show that $f$ is neither injective nor surjective.

(Not injective) Consider $1,-1\in\Z$. We have $f(1)=1^2+1=2$ and $f(-1)=(-1)^2+1=2$, thus $f(1)=f(-1)$. However, $1\neq -1$. Therefore, $f$ is not injective.

(Not surjective) Consider $0\in\Z$. Assume, for the sake of contradiction, that there exists $x\in\Z$ such that $f(x)=0$. Then $x^2+1=0$, so $x^2=-1$. But since $x\in\Z$, we have $x^2>0$. This is a contradiction. Thus our assumption is false: For $0\in\Z$, there does not exist $x\in\Z$ such that $f(x)=0$. Therefore, $f$ is not surjective.

Hence, $f$ is neither injective nor surjective.$~\square$
\item\textit{Claim. }Let $f:\Z\to\Z$ with 
\[f(n) = \begin{cases} 
\frac{n}{2}&n\text{ is even}\\
0&n\text{ is odd}
\end{cases}
\]
for $n\in\Z$. Then $f$ is not one-to-one, but $f$ is onto.\newline

\textit{Proof. }Let $f:\Z\to\Z$ with 
\[f(n) = \begin{cases} 
\frac{n}{2}&n\text{ is even}\\
0&n\text{ is odd}
\end{cases}
\]
for $n\in\Z$. We want to show that $f$ is not injective but surjective.

(Not injective) Consider $f(1)$ and $f(3)$. We have $f(1)=f(3)=0$. However, $1\neq 3$. Therefore, $f$ is not injective.

(Surjective) Let $y\in\Z$. Note $2y\in\Z$ and $2y$ is even. So there exists $2y\in\Z$ such that $f(2y)=y$. Therefore, $f$ is surjective.

Hence, $f$ is not injective, but $f$ is surjective.$~\square$
\item\textit{Claim. }Let $f:\R\to\R$ with 
\[f(n) = \begin{cases} 
\frac{1}{x}&x\neq 0\\
0&x=0
\end{cases}
\]
for $x\in\R$. Then $f$ is one-to-one and $f$ is onto.\newline

\textit{Proof. }Let $f:\R\to\R$ with 
\[f(n) = \begin{cases} 
\frac{1}{x}&x\neq 0\\
0&x=0
\end{cases}
\]
for $x\in\R$. We want to show that $f$ is bijective.\newline

(Injective) Let $x_1,x_2\in\R$. Suppose $f(x_1)=f(x_2)$. Then $f(x_1)=f(x_2)=0$ or $\left(f(x_1)=f(x_2)\right)\neq 0$.

Suppose $f(x_1)=f(x_2)=0$. We know $\frac{1}{x}\neq 0$ for all $x\in\R$, so $x_1=x_2=0$.

Suppose $\left(f(x_1)=f(x_2)\right)\neq 0$. Then $\frac{1}{x_1}=\frac{1}{x_2}$, and $x_1\neq 0$ and $x_2\neq 0$, so $x_1=x_2$.

In all cases, if $f(x_1)=f(x_2)$, then $x_1=x_2$. Therefore $f$ is injective.\newline

(Surjective) Let $y\in\R$. Then $y=0$ or $y\neq 0$.

Suppose $y=0$. Consider $0\in\R$. We have $y=f(0)=0$. There exists $0\in\R$ such that $y=f(0)$.

Suppose $y\neq 0$. Note $\frac{1}{y}\in\R$ and $\frac{1}{y}\neq 0$. So there exists $\frac{1}{y}\in\Z$ such that $\frac{1}{y}\neq 0$ and $f\left(\frac{1}{y}\right)=y$.

In all cases, $f$ is surjective. Therefore, $f$ is surjective.\newline

Hence, $f$ is bijective.$~\square$
\item\textit{Claim. }Let $f:\N\to\N$ with 
\[f(n) = \begin{cases} 
2^n&n\text{ is even}\\
n&n\text{ is odd}
\end{cases}
\]
for $n\in\N$. Then $f$ is neither one-to-one nor onto.\newline

\textit{Proof. }Let $f:\N\to\N$ with 
\[f(n) = \begin{cases} 
2^n&n\text{ is even}\\
n&n\text{ is odd}
\end{cases}
\]
for $x\in\N$. We want to show that $f$ is not injective and not surjective.\newline

(Not injective) Consider $0,1\in\N$. We have $f(0)=2^0=1$ and $f(1)=1$, thus $f(0)=f(1)$. However $0\neq 1$. Therefore, $f$ is not injective.\newline

(Not surjective) Consider $6\in\N$. Assume, for the sake of contradiction, there exists $x\in\N$ such that $f(x)=6$. Then $x$ is even or $x$ is odd.

Suppose $x$ is odd. Then $f(x)=x=6$. Then there exists $k_1\in\Z$ such that $x=2k_1-1$. So $2k_1=7$, so $k_1=\frac{7}{2}$. But $k_1$ is an integer. This is a contradiction.

Suppose $x$ is even. Then there exists $k_2\in\Z$ such that $x=2k_2$. So $f(2k_2)=2^{2k_2}$. Thus $6=4^{k_2}$, so $k_2=\log_4{6}$. Note $3<k_2<4$. But $k_2$ is an integer. This is a contradiction.

In all cases, we arrive at a contradiction. Thus our assumption is false: For $6\in\N$, there does not exist $x\in\N$ such that $f(x)=6$. Therefore, $f$ is not surjective.\newline

Hence, $f$ is neither injective nor surjective.$~\square$
\item\textit{Claim. }Let $f:\mathcal{P}(\Z)\to\mathcal{P}(\Z)$ with $f(A)=A\cup\{0\}$ for $A\in\mathcal{P}(\Z)$. Then $f$ is not one-to-one and $f$ is not onto.\newline

\textit{Proof. }Let $f:\mathcal{P}(\Z)\to\mathcal{P}(\Z)$ with $f(A)=A\cup\{0\}$ for $A\in\mathcal{P}(\Z)$. Then $f$ is neither injective nor surjective.\newline

(Not injective) Consider $\{1\},\{0,1\}\in\mathcal{P}(\Z)$. We have $f(\{1\})=\{1\}\cup\{0\}=\{0,1\}$ and $f(\{0,1\})=\{0,1\}\cup\{0\}=\{0,1\}$, thus $f(\{1\})=f(\{0,1\})$. However, $\{1\}\neq\{0,1\}$. Therefore, $f$ is not injective.\newline

(Not surjective) Consider $\{1\}\in\mathcal{P}(\Z)$. Assume, for the sake of contradiction, that there exists $X\in\mathcal{P}(\Z)$ such that $f(X)=\{1\}$. But $f(X)=X\cup\{0\}$, so $\{1\}=X\cup\{0\}$. Of course $0\in(X\cup\{0\})$, but $0\notin\{1\}$. This is a contradiction. Thus our assumption is false: For $\{1\}\in\mathcal{P}(\Z)$, there does not exist $X\in\mathcal{P}(\Z)$ such that $f(X)=\{1\}$. Therefore, $f$ is not surjective.\newline

Hence, $f$ is neither injective nor surjective.$~\square$
\end{enumerate}
\end{solution}
\begin{question}
    \begin{enumerate}
        \item Let $A = \{1,2,3,4\}$ and $B = \{5,6,7\}$.
        Let $f$ be the relation $ \left\{(1,5),(2,5),(3,6),(?,?)\right\} $ where the question marks are to be filled in by you. 
        Give an example of $(?,?) \in A \times B$ so that: (Remember to explain your reasoning.)
            \begin{enumerate}
                \item The relation $f$ is not a function.
                \item The relation is a function from $A$ to $B$ but not onto $B$.
                \item The relation is a function from $A$ to $B$ and is onto $B$.
            \end{enumerate}

        \item     Let $A$ be an $n$-element set and let $i, j, k \in \mathbb{N}$ with $i+j+k = n$.
        How many functions $f \colon A \to \{0,1,2\}$ are there for which all three of the below are satisfied:
            \begin{itemize}
                \item $\left|\left\{ a \in A : f(a) = 0 \right\} \right| = i$
                \item $\left|\left\{ a \in A : f(a) = 1 \right\} \right| = j$
                \item $\left|\left\{ a \in A : f(a) = 2 \right\} \right| = k$
            \end{itemize}
    \end{enumerate}
\end{question}
\begin{solution}
\begin{enumerate}
    \item\begin{enumerate}
    \item Let $f$ be a relation where \[f=\{(1,5),(2,5),(3,6),(1,6)\}.\] Then $f$ is not a function because $1~f~5$ and $1~f~6$ and $5\neq 6$.
    \item Let $f$ be a relation where \[f=\{(1,5),(2,5),(3,6),(4,6)\}.\] Then $f$ is a function because $(a,b)\in f$ and $(a,c)\in f$ implies $b=c$. In other words, for every input, there is exactly one output. We know $f(1)=5,f(2)=5,f(3)=6,$ and $f(4)=6$. So there are no repeated inputs with different outputs. Now \[\dom{f}=\{1,2,3,4\}=A.\] \[\im{f}=\{5,6\}\subseteq B.\] Since $f$ is a function, $\dom{f}=A$, and $\im{f}\subseteq B$, we have $f:A\to B$. Since $\im{f}\neq B$, we know $f$ is not surjective (onto $B$).
    \item Let $f$ be a relation where \[f=\{(1,5),(2,5),(3,6),(4,7)\}.\] Then $f$ is a function because $(a,b)\in f$ and $(a,c)\in f$ implies $b=c$. In other words, for every input, there is exactly one output. We know $f(1)=5,f(2)=5,f(3)=6,$ and $f(4)=7$. So there are no repeated inputs with different outputs. Now \[\dom{f}=\{1,2,3,4\}=A.\] \[\im{f}=\{5,6,7\}=B.\] Since $f$ is a function, $\dom{f}=A$, and $\im{f}\subseteq B$, we have $f:A\to B$. Since $\im{f}=B$, we know $f$ is surjective (onto $B$).
    \end{enumerate}
    \item We want to count the number of functions $f:A\to\{0,1,2\}$. Since $\dom{f}=A$ and $|A|=n$, there are $n$ choices for each input and all $n$ elements must be used. Since $f$ is a function, repetition of inputs is not allowed. So there are $n!$ ways to construct the domain of $f$, holding the ordering of the codomain constant. We have $i$ instances of 0, $j$ instances of 1, and $k$ instances of 2 in the image. We have ``double-counted'' functions. For each unique domain constructed, there are $i!$ functions with identical images due to differences in placement of identical ``0''s, $j!$ functions with identical images due to rearrangements of ``1''s, and $k!$ functions with identical images due to rearrangements of ``2''. So we divide $n!$ by $i!,j!$, and $k!$, yielding $\frac{n!}{i!j!k!}$ possible functions that satisfy the requirements.
\end{enumerate}
\end{solution}
\begin{question}
    \begin{enumerate}
	\item There are five points inside an equilateral triangle of side length 2. 
	Show that at least two of the points are within 1 unit distance from each other.
	\item Let $A$ be a set of 10 distinct integers between 1 and 100 (both inclusive). 
	Show that there are two nonempty and disjoint subsets of $A$ such that the sum of all its elements are the same.
    \end{enumerate}
\end{question}
\begin{solution}
\begin{enumerate}
\item Let $A=(0,0)$, $B=(2,0)$, and $C=(1,\sqrt{3})$.

\noindent If $A,B,$ and $C$ represent points in a plane, then the distance from $A$ to $B$ is $|2-0|=2$ units, the distance from $B$ to $C$ is $\sqrt{(1-2)^2+(\sqrt{3}-0)^2}=2$ units, and the distance from $C$ to $A$ is $\sqrt{1^2+\sqrt{3}^2}=2$ units. So $\{A,B,C\}$ is the set of vertices of an equilateral triangle with side lengths of 2 units.

\noindent Let $a=\left(\frac{2+1}{2},\frac{0+\sqrt{3}}{2}\right)=\left(\frac{3}{2},\frac{\sqrt{3}}{2}\right)$ be the midpoint of the line segment from $B$ to $C$, $b=\left(\frac{1}{2},\frac{\sqrt{3}}{2}\right)$ be the midpoint of the line segment from $A$ to $C$, and $c=\left(1,0\right)$ be the midpoint of the line segment from $A$ to $B$. Now we can construct 4 more equilateral triangles, each denoted by a set of vertices:
\begin{align}
\{a,b,c\}\\
\{A,b,c\}\\
\{a,B,c\}\\
\{a,b,C\}
\end{align}
By construction, the side lengths of each of these triangles are 1 unit. Suppose there are 5 points within the outer triangle bounded by $A,B$, and $C$. Each of the 5 points (``pigeons'') is in one of 4 inner triangles (``pigeonholes'') listed above. Since $4<5$, at least 2 points lie within the same inner triangle by the pigeonhole principle. Since the vertices of each inner triangle are 1 unit apart, the maximum distance between 2 points in the same inner triangle is 1 unit.

\noindent Hence, at least 2 of the 5 points lie within 1 unit of one another.$~\square$
\item Let $A=\{a:a\in\Z,1\leq a\leq 100\}$ such that $|A|=10$. Then $2^A$ contains all subsets of $A$, including the empty set.

Consider $A'\subseteq A$.

Note if $A'=\{1\}$, then $\sum_{a'\in A'}{a'}=1$. By construction, this set is the value of $A'$ with the minimum sum of all elements.

If $A'=\{91,92,93,94,95,96,97,98,99,100\}$, then $\sum_{a'\in A'}{a'}=955$. By construction, this set is the value of $A'$ with the maximum sum of all elements.

We can construct a value of $A'$ such that the sum of all elements in $A'$ is equal to any natural number from 1 (inclusive) to 955 (inclusive), so there are 955 possible sums of the elements in $A'$ (``pigeonholes'').

Let $n$ be the number of nonempty subsets of $A$. Then
\begin{align*}
n
&=\left|2^A\right|-1\\
&=2^{|A|}-1\\
&=2^{10}-1\\
&=1024-1\\
&=1023.
\end{align*}
There are $n=1023$ nonempty subsets of $A$ (``pigeons''). Note $n>955$. By the pigeonhole principle, there are at least 2 distinct nonempty subsets of $A$ whose elements have the same sum.

So there exist sets $P$ and $Q$ such that all of the following are true:
\begin{itemize}
    \item$P\neq Q$,
    \item$P\neq\emptyset$,
    \item$Q\neq\emptyset$,
    \item$P\subseteq A$,
    \item$Q\subseteq A$, and
    \item$\sum_{p\in P}{p}=\sum_{q\in Q}{q}.$
\end{itemize}
Then either $P\cap Q=\emptyset$ or $P\cap Q\neq\emptyset$.

Suppose $P\cap Q=\emptyset$. Therefore, there exist 2 nonempty and disjoint subsets of $A$ such that the sum of all elements are the same.

Suppose $P\cap Q\neq\emptyset$. Let $P'=P-Q$ and $Q'=Q-P$. Since $P\neq Q$, we have $P'\neq Q'$, $P'\neq\emptyset$, and $Q'\neq\emptyset$. Since $P'\subseteq P$, we have $P'\subseteq A$. Since $Q'\subseteq Q$, we have $Q'\subseteq A$.

Furthermore, since $P'=P-(P\cap Q)$ and $Q'=Q-(Q\cap P)$, we know \[\sum_{p'\in P'}{p'}=\left(\sum_{p\in P}{p}\right)-\left(\sum_{x\in P\cap Q}{x}\right)\] and \[\sum_{q'\in Q'}{q'}=\left(\sum_{q\in Q}{q}\right)-\left(\sum_{x\in P\cap Q}{x}\right).\] Since $\sum_{p\in P}{p}=\sum_{q\in Q}{q}$, we have $\sum_{p'\in P'}{p'}=\sum_{q'\in Q'}{q'}$.

By construction, $P'\cap Q'=\emptyset$. Therefore, there exist 2 nonempty and disjoint subsets of $A$ such that the sum of all elements are the same.

In all cases, the claim is true. Hence, in a set of 10 distinct integers between 1 (inclusive) and 100 (inclusive), there are 2 nonempty and disjoint subsets such that the sums of all the elements in each subset are equal.$~\square$
\end{enumerate}
\end{solution}
\begin{question}
    Let $A = \{ x \in \Z ~:~ 3|x \}$.  Show that $A$ and $\mathbb{N}$ has the same cardinality.  {\it Hint:} Define $f: A \rightarrow \N$ such that
\[ f(x) =\left\{\begin{array}{c c l} \frac{2}{3}x & & \text{if } x \geq 0 \\ ~ \\ -\frac{2}{3}x - 1 &  & \text{if } x < 0 \end{array} \right. .\]
\end{question}
\begin{solution}\newline

\noindent\textit{Claim. }Let $A=\{x\in\Z:3\mid x\}$. Then $A$ and $\N$ have the same cardinality.
\begin{proof}
Let $A=\{x\in\Z:3\mid x\}$. We will demonstrate the existence of a one-to-one correspondence between $A$ and $\N$.\newline

\noindent Let $f:A\to\N$ such that
\[f(a) = \begin{cases} 
\frac{2}{3}a&a\geq 0\\
-\frac{2}{3}a-1&a<0
\end{cases}
\]
for $a\in A$. We want to demonstrate that $f$ is bijective, requiring $f$ is injective and $f$ is surjective.\newline

\noindent(Injective) Let $a_1,a_2\in A$. Suppose $f(a_1)=f(a_2)$. Then one of the following is true:
\begin{itemize}
    \item $a_1\geq 0$ and $a_2\geq 0$, or
    \item $a_1<0$ and $a_2<0$, or
    \item $a_1\geq 0$ and $a_2<0$ (symmetrically, this is the same as $a_2\geq 0$ and $a_1<0$).
\end{itemize}

Suppose $a_1\geq 0$ and $a_2\geq 0$. Since $f(a_1)=f(a_2)$, we have $\frac{2}{3}a_1=\frac{2}{3}a_2$. So $a_1=a_2$.\newline

Suppose $a_1<0$ and $a_2<0$. Since $f(a_1)=f(a_2)$, we have $\left(-\frac{2}{3}a_1-1\right)=\left(-\frac{2}{3}a_2-1\right)$. Again $\frac{2}{3}a_1=\frac{2}{3}a_2$. So $a_1=a_2$.\newline

Assume, for the sake of contradiction, $a_1\geq 0$ and $a_2<0$. Note $a_1\in A$, so $3\mid a_1$. Note also $a_2\in A$ and $a_2<0$, so $3\mid a_2$ and $(-1)\mid a_2$. So there exist $n_1,n_2\in\Z$ such that $a_1=3n_1$ and $a_2=-3n_2$. Observe
\begin{align*}
f(a_1)&=f(a_2)\\
\frac{2}{3}a_1&=-\frac{2}{3}a_2-1\\
\frac{2}{3}(3n_1)&=-\frac{2}{3}(-3n_2)-1\\
2n_1&=2n_2-1.
\end{align*}

Note $2\mid n_1$. By definition, $2n_1$ is even and $2n_2-1$ is odd, so these quantities cannot be equal. We have arrived at a contradiction. Thus our assumption is false: It is not the case that $a_1\geq 0$ and $a_2<0$. The argument is symmetrical for the case where $a_2\geq 0$ and $a_1<0$. So it is not the case that $a_2\geq 0$ and $a_1<0$ either.\newline

In all possible cases, we have $a_1=a_2$. Therefore, $f:A\to\N$ is injective.\newline

\noindent(Surjective) Let $y\in\N$. Then $y$ is even or $y$ is odd.\newline

Suppose $y$ is even. Then there exists $m_1\in\Z$ such that $y=2m_1$. Consider $\frac{3}{2}y=3m_1$. Note $3\mid\left(\frac{3}{2}y\right)$, so $\frac{3}{2}y\in A$. Of course, \[f\left(\frac{3}{2}y\right)=\left(\frac{2}{3}\right)\left(\frac{3}{2}y\right)=y\] so there exists $\frac{3}{2}y\in A$ such that $f\left(\frac{3}{2}y\right)=y$.\newline

Suppose $y$ is odd. Then there exists $m_2\in\Z$ such that $y=2m_2-1$. Consider $-\frac{3}{2}y-\frac{3}{2}=-3m_2$. Since $-\frac{3}{2}y-\frac{3}{2}=3(-m_2)$, we have $3\mid\left(-\frac{3}{2}y-\frac{3}{2}\right)$, so $-\frac{3}{2}y-\frac{3}{2}\in A$. Of course, \[f\left(-\frac{3}{2}y-\frac{3}{2}\right)=-\left(\frac{2}{3}\right)\left(-\frac{3}{2}y-\frac{3}{2}\right)-1=y\] so there exists $-\frac{3}{2}y-\frac{3}{2}\in A$ such that $f\left(-\frac{3}{2}y-\frac{3}{2}\right)=y$.\newline

In both cases, $f$ is surjective. Therefore, $f:A\to\N$ is surjective.\newline

\noindent Hence, $f:A\to\N$ is bijective. Ergo $A$ and $\N$ have the same cardinality.
\end{proof}
\end{solution}
\end{document}