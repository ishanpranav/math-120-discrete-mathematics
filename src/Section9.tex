\documentclass[12pt]{article}
\usepackage[english]{babel}
\usepackage[letterpaper,top=2cm,bottom=2cm,left=3cm,right=3cm,marginparwidth=1.75cm]{geometry}
\usepackage{amsmath}
\usepackage{amsfonts}
\usepackage{graphicx}
\title{MATH-UA 120 Section 9}
\author{Ishan Pranav}
\date{September 16, 2023}
\begin{document}
\maketitle
\section*{Factorial}
The quantity
\[(n)_n=n(n-1)(n-2)(n-3)\dots(n-n+1)=(n)(n-1)(n-2)\dots(1)\]
occurs frequently in mathematics and has a special name and notation; it is called $n$ \textit{factorial} and is written $n!$.

\[0!=1.\]

Formally,
\[n!=\prod_{k=1}^{n}{k}.\]
\end{document}