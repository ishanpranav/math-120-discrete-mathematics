\documentclass[12pt]{article}
\usepackage[english]{babel}
\usepackage[letterpaper,top=2cm,bottom=2cm,left=3cm,right=3cm,marginparwidth=1.75cm]{geometry}
\usepackage{amsfonts}
\usepackage{amsmath}
\usepackage{graphicx}
\title{MATH-UA 120 Section 3}
\author{Ishan Pranav}
\date{September 1, 2023}
\begin{document}
\maketitle
\section*{Integer}
The \textit{integers} are the positive whole numbers, the negative whole numbers, and zero. That is, the set of integers, denoted by the letter $\mathbb{Z}$, is
\[\mathbb{Z}=\{\dots,-3,-2,-1,0,1,2,3,\dots\}.\]
\section*{Divisible}
Let $a$ and $b$ be integers. We say that $a$ is \textit{divisible} by $b$ provided there is an integer $c$ such that $bc=a$. We also say $b$ \textit{divides} $a$, or $b$ is a \textit{factor} of $a$, or $b$ is a \textit{divisor} of $a$. The notation for this is $b|a$.
\section*{Even}
An integer is called \textit{even} provided it is divisible by two.
\section*{Odd}
An integer $a$ is called \textit{odd} provided there is an integer $x$ such that $a=2x+1$.
\section*{Prime}
An integer $p$ is called \textit{prime} provided that $p>1$ and the only positive divisors of $p$ are 1 and $p$.
\section*{Composite}
A positive integer $a$ is called \textit{composite} provided there is an integer $b$ such that $1<b<a$ and $b|a$.
\end{document}