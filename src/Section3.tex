\documentclass[12pt]{article}
\usepackage[english]{babel}
\usepackage[letterpaper,top=2cm,bottom=2cm,left=3cm,right=3cm,marginparwidth=1.75cm]{geometry}
\usepackage{amsfonts}
\usepackage{amsmath}
\usepackage{amssymb}
\usepackage{graphicx}
\renewcommand{\labelenumi}{\alph{enumi}.}
\title{MATH-UA 120 Section 3}
\author{Ishan Pranav}
\date{September 1, 2023}
\begin{document}
\maketitle
\section*{Integer}
The \textit{integers} are the positive whole numbers, the negative whole numbers, and zero. That is, the set of integers, denoted by the letter $\mathbb{Z}$, is
\[\mathbb{Z}=\{\dots,-3,-2,-1,0,1,2,3,\dots\}.\]
\section*{Divisible}
Let $a$ and $b$ be integers. We say that $a$ is \textit{divisible} by $b$ provided there is an integer $c$ such that $bc=a$. We also say $b$ \textit{divides} $a$, or $b$ is a \textit{factor} of $a$, or $b$ is a \textit{divisor} of $a$. The notation for this is $b|a$.
\section*{Even}
An integer is called \textit{even} provided it is divisible by two.
\section*{Odd}
An integer $a$ is called \textit{odd} provided there is an integer $x$ such that $a=2x+1$.
\section*{Prime}
An integer $p$ is called \textit{prime} provided that $p>1$ and the only positive divisors of $p$ are 1 and $p$.
\section*{Composite}
A positive integer $a$ is called \textit{composite} provided there is an integer $b$ such that $1<b<a$ and $b|a$.
\section*{Natural number}
The set of \textit{natural numbers} (the nonnegative integers), denoted by the letter $\mathbb{N}$, is 
\[\mathbb{N}=\{0,1,2,3,\dots\}.\]
\section*{Rational number}
The set of \textit{rational numbers} (the numbers formed by dividing two integers), denoted by the letter $\mathbb{Q}$, is
\[\mathbb{Q}=\left\{\frac{a}{b}:\,a,b\in\mathbb{Z}\text{ and }b\neq 0\right\}.\]
\section*{Perfect}
An integer $n$ is called \textit{perfect} provided it equals the sum of all its divisors that are both positive and less than $n$. For example, 28 is perfect because the positive divisors of 28 are 1, 2, 4, 7, 14, and 28. Note that $1+2+4+7+14=28$.
\section{Determine}
Please determine which of the following are true and which are false.
\begin{enumerate}
    \item$3\nmid 100$. We want to find an integer $c$ such that $100=3c$. No such $c\in\mathbb{Z}$ exists. Therefore, $3\nmid 100$.
    \item$3 \mid 99$. We want to find an integer $c$ such that $99=3c$. Let $c=33$. Thus, $99=3\times 33$. Therefore, $3 \mid 99$.
    \item$-3 \mid 3$. We want to find an integer $c$ such that $3=-3c$. Let $c=-1$. Thus, $3=-3\times-1$. Therefore, $-3 \mid 3.$
    \item$-5 \mid -5$. We want to find an integer $c$ such that $-5=-5c$. Let $c=1$. Thus, $-5=-5\times 1$. Therefore, $-5 \mid -5.$
    \item$-2\nmid-7$. We want to find an integer $c$ such that $-7=-2c$. No such $c\in\mathbb{Z}$ exists. Therefore $-2\nmid -7$.
    \item$0\nmid 4$. We want to find an integer $c$ such that $4=0c$, or $4=0$. This statement is absurd. Therefore, $0\nmid 4$.
    \item$4 \mid 0$. We want to find an integer $c$ such that $0=4c$. Let $c=0$. Thus, $0=4\times 0$. Therefore, $4 \mid 0$.
    \item$0 \mid 0$. We want to find an integer $c$ such that $0=0c$. Let $c=1$. Thus, $0=0\times 1$. Therefore, by our definition of \textit{divisible}, $0 \mid 0$.
\end{enumerate}
\section{A possible alternative}
The alternative definition of \textit{divisible} states that ``$a$ is divisible by $b$ provided $\frac{a}{b}$ is an integer.'' This alternative definition is different from the definition above because it involves division instead of multiplication. Division enforces an additional restriction on the inputs: The divisor must not be zero. According to the original definition, $0 \mid 0$ because an integer $c$ exists such that $0=0c$ (for example, letting $c=1$ yields $0=0\times 1$). However, the alternative definition disagrees: $0\nmid 0$ because $\frac{0}{0}$ is undefined.
\section{None}
None of the following numbers is prime. Explain why. Which of the following are composite?
\begin{enumerate}
    \item 21 is composite. To demonstrate that 21 is prime, want to show that the only positive divisors of 21 are 1 and 21 and that 21 is greater than 1. $21>1$ and $3 \mid 21$. To illustrate that $3 \mid 1$, we must find an integer $c$ such that $21=3c$. Let $c=7$. Thus, $21=3\times 7$. Therefore, $3 \mid 21$. However, 3 is neither 1 nor 21, so 21 is not prime. For 21 to be composite, there must be an integer $b$ such that $1<b<21$ and $b \mid 21$. Let $b=3$. $1<3<21$ and $3 \mid 21$, so 21 is composite.
    \item 0 is not prime. By definition, prime numbers must be greater than 1. $0<1$. Therefore, 0 is not prime. For 0 to be composite, there must be an integer $b$ such that $1<b<0$, which is absurd. Therefore, 0 is not composite.
    \item$\pi$ is not prime. By definition, prime and composite numbers must be integers. $\pi\notin\mathbb{Z}$. Therefore, $\pi$ is neither prime nor composite.
    \item$\frac{1}{2}$ is not prime. By definition, prime and composite numbers must be integers. $\pi\notin\mathbb{Z}$. Therefore, $\pi$ is neither prime nor composite.
    \item-2 is not prime. By definition, prime numbers must be greater than 1. $-2<1$. Therefore -2 is not prime. For -2 to be composite, there must be an integer $b$ such that $1<b<-2$, which is absurd. Therefore, -2 is not composite.
    \item-1 is not prime. By definition, prime numbers must be greater than 1. $-1<1$. Therefore -2 is not prime. For -1 to be composite, there must be an integer $b$ such that $1<b<-1$, which is absurd.
\end{enumerate}
\section{Create definitions}
\subsection{Less than or equal to}
Let $x$ and $y$ be integers. We say that $x$ is \textit{less than or equal to} $y$ (denoted $x\leq y$) provided $y-x\in\mathbb{N}$.
\subsection{Less than}
Let $x$ and $y$ be integers. We say that $x$ is \textit{less than} $y$ (denoted $x<y$) provided $x\leq y$ and $x\neq y$.
\subsection{Greater than or equal to}
Let $x$ and $y$ be integers. We say that $x$ is \textit{greater than or equal to} $y$ (denoted $x\geq y$) provided that $x$ is not less than $y$.
\subsection{Greater than}
Let $x$ and $y$ be integers. We say that $x$ is \textit{greater than} $y$ (denoted $x>y$) provided that $x$ is not less than or equal to $y$.
\section{Explain}
Every integer is a rational number, but not every rational number is an integer because every integer $k$ can be expressed as a ratio $\frac{k}{1}$, but there are rational numbers that are not integers, such as $\frac{1}{2}.$
\section{Define perfect square}
An integer $x$ is called a \textit{perfect square} provided there is an integer $y$ such that $y^2=x$.
\section{Define square root}
A number $x$ is a \textit{square root} of a number $y$ provided that $x^2=y$.
\section{Define perimeter of a polygon}
The \textit{perimeter} of a polygon is the total length of its boundary.
\section{Define between}
\section{Define midpoint of a line segment}
The \textit{midpoint} of a line segment $\overline{AB}$ is a point $C$ on the segment such that the distance from $A$ to $C$ equals the distance from $C$ to $B$.
\section{Try writing definitions}
\subsection{Teenager}
A person $p$ is called a \textit{teenager} provided that $p$'s age is between thirteen and nineteen.
\subsection{Grandmother}
Person $A$ is the \textit{grandmother} of person $B$ provided $A$ is female and $A$ is the parent of one of $B$'s parents.
\subsection{Leap year}
A year $Y$ with the number $y$ in the Gregorian Calendar is called a \textit{leap year} provided $y$ is divisible by four hundred or else divisible by four but not one hundred.
\subsection{Dime}
A coin $c$ is called a \textit{dime} provided that its face value is five cents.
\subsection{Palindrome}
A word $w$ is called a \textit{palindrome} provided its spelling is identical when characters are read from left to right as when read from right to left.
\subsection{Homophone}
A pair of words, $w$ and $x$, are called \textit{homophones} if they are spelled differently but share the same pronunciation.
\subsection{Counting problems}
\begin{enumerate}
    \item 8 has four positive divisors: 1, 2, 4, 8.
    \item 32 has six positive divisors: 1, 2, 4, 8, 16, 32.
    \item$2^n$ has $n+1$ positive divisors, where $n$ is a positive integer.
    \item 10 has 4 positive divisors: 1, 2, 5, 10.
    \item 100 has 9 positive divisors: 1, 2, 4, 5, 10, 20, 25, 50, 100.
    \item 1,000,000 has 49 positive divisors.
    \item$10^n$ has $n^2$ positive divisors, where $n$ is a positive integer.
    \item$30=2\times 3\times 5$ has 8 positive divisors: 1, 2, 3, 5, 6, 10, 15, 30.
    \item$42=2\times 3\times 7$ has 8 positive divisors: 1, 2, 3, 6, 7, 14, 21, 42. Thirty and forty-two have the same number of positive divisors because the product of the counts of the prime factors, adding one to each, is the same: $2\times 2\times 2 = 2\times 2\times 2 = 8$.
    \item$2310=2\times 3\times 5\times 7\times 11$ has $2^5=32$ factors.
    \item$1\times 2\times 3\times 4\times 5\times 6\times 7\times 8$ can be expressed as $2^7\times 3^2\times 5\times 7$. Thus it has $8\times 3\times 2^2=96$ factors.
    \item 0 has uncountably many positive divisors.
\end{enumerate}
\section{Find}
\begin{enumerate}
    \item There is a perfect number smaller than 28: 6 is perfect because the positive divisors of 6 are 1, 2, 3, and 6. Note that $1+2+3=6$.
    \item A computer program to find the next perfect number after 28:
\begin{verbatim}
#include <limits.h>
#include <stdio.h>
\end{verbatim}

\verb|/**|

\verb| * | Computes the first perfect number greater than the given number.

\verb| * |

\verb| * @param | min the previous perfect number from which to begin the search

\verb| * @return |The next perfect number.

\verb|*/|
\begin{verbatim}
static int getPerfectAbove(int min)
{
   for (int value = min + 1; value < INT_MAX; value++)
   {
      int sum = 0;

      for (int factor = 1; factor < value; factor++)
      {
         if (value % factor == 0)
         {
            sum += factor;   
         }
      }
        
      if (value == sum)
      {
         return value;
      }
    }
}
\end{verbatim}

\verb|/**|

\verb| * |The main entry point for the application.

\verb| * |

\verb| * @return |An exit code.

\verb|*/|

\begin{verbatim}
int main()
{
   printf("%d", getPerfectAbove(28));

   return 0;
}
\end{verbatim}
\end{enumerate}
\section{The mathematician}
Mathematicians rely on definitions. In the story, the mathematician does not deviate from the definition.
\end{document}
