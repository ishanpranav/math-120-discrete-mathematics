\documentclass[12pt]{article}
\usepackage[english]{babel}
\usepackage[letterpaper,top=2cm,bottom=2cm,left=3cm,right=3cm,marginparwidth=1.75cm]{geometry}
\usepackage{amsfonts}
\usepackage{amsmath}
\usepackage{amssymb}
\usepackage{graphicx}
\renewcommand{\labelenumi}{\alph{enumi}.}
\title{MATH-UA 120 Section 3}
\author{Ishan Pranav}
\date{September 1, 2023}
\begin{document}
\maketitle
\section*{Integer}
The \textit{integers} are the positive whole numbers, the negative whole numbers, and zero. That is, the set of integers, denoted by the letter $\mathbb{Z}$, is
\[\mathbb{Z}=\{\dots,-3,-2,-1,0,1,2,3,\dots\}.\]
\section*{Divisible}
Let $a$ and $b$ be integers. We say that $a$ is \textit{divisible} by $b$ provided there is an integer $c$ such that $bc=a$. We also say $b$ \textit{divides} $a$, or $b$ is a \textit{factor} of $a$, or $b$ is a \textit{divisor} of $a$. The notation for this is $b|a$.
\section*{Even}
An integer is called \textit{even} provided it is divisible by two.
\section*{Odd}
An integer $a$ is called \textit{odd} provided there is an integer $x$ such that $a=2x+1$.
\section*{Prime}
An integer $p$ is called \textit{prime} provided that $p>1$ and the only positive divisors of $p$ are 1 and $p$.
\section*{Composite}
A positive integer $a$ is called \textit{composite} provided there is an integer $b$ such that $1<b<a$ and $b|a$.
\section*{Natural number}
The set of \textit{natural numbers} (the nonnegative integers), denoted by the letter $\mathbb{N}$, is 
\[\mathbb{N}=\{0,1,2,3,\dots\}.\]
\section*{Rational number}
The set of \textit{rational numbers} (the numbers formed by dividing two integers), denoted by the letter $\mathbb{Q}$, is
\[\mathbb{Q}=\left\{\frac{a}{b}:\,a,b\in\mathbb{Z}\text{ and }b\neq 0\right\}.\]
\section{Determine}
Please determine which of the following are true and which are false.
\begin{enumerate}
    \item$3\nmid 100$. We want to find an integer $c$ such that $100=3c$. No such $c\in\mathbb{Z}$ exists. Therefore, $3\nmid 100$.
    \item$3\,|\,99$. We want to find an integer $c$ such that $99=3c$. Let $c=33$. Thus, $99=3\times 33$. Therefore, $3\,|\,99$.
    \item$-3\,|\,3$. We want to find an integer $c$ such that $3=-3c$. Let $c=-1$. Thus, $3=-3\times-1$. Therefore, $-3\,|\,3.$
    \item$-5\,|\,-5$. We want to find an integer $c$ such that $-5=-5c$. Let $c=1$. Thus, $-5=-5\times 1$. Therefore, $-5\,|\,-5.$
    \item$-2\nmid-7$. We want to find an integer $c$ such that $-7=-2c$. No such $c\in\mathbb{Z}$ exists. Therefore $-2\nmid -7$.
    \item$0\nmid 4$. We want to find an integer $c$ such that $4=0c$, or $4=0$. This statement is absurd. Therefore, $0\nmid 4$.
    \item$4\,|\,0$. We want to find an integer $c$ such that $0=4c$. Let $c=0$. Thus, $0=4\times 0$. Therefore, $4\,|\,0$.
    \item$0\,|\,0$. We want to find an integer $c$ such that $0=0c$. Let $c=1$. Thus, $0=0\times 1$. Therefore, by our definition of \textit{divisible}, $0\,|\,0$.
\end{enumerate}
\section{A possible alternative}
The alternative definition of \textit{divisible} states that ``$a$ is divisible by $b$ provided $\frac{a}{b}$ is an integer.'' This alternative definition is different from the definition above because it involves division instead of multiplication. Division enforces an additional restriction on the inputs: The divisor must not be zero. According to the original definition, $0\,|\,0$ because an integer $c$ exists such that $0=0c$ (for example, letting $c=1$ yields $0=0\times 1$). However, the alternative definition disagrees: $0\nmid 0$ because $\frac{0}{0}$ is undefined.
\section{None}
None of the following numbers is prime. Explain why. Which of the following are composite?
\begin{enumerate}
    \item 21 is composite. To demonstrate that 21 is prime, want to show that the only positive divisors of 21 are 1 and 21 and that 21 is greater than 1. $21>1$ and $3\,|\,21$. To illustrate that $3\,|\,1$, we must find an integer $c$ such that $21=3c$. Let $c=7$. Thus, $21=3\times 7$. Therefore, $3\,|\,21$. However, 3 is neither 1 nor 21, so 21 is not prime. For 21 to be composite, there must be an integer $b$ such that $1<b<21$ and $b\,|\,21$. Let $b=3$. $1<3<21$ and $3\,|\,21$, so 21 is composite.
    \item 0 is not prime. By definition, prime numbers must be greater than 1. $0<1$. Therefore, 0 is not prime. For 0 to be composite, there must be an integer $b$ such that $1<b<0$, which is absurd. Therefore, 0 is not composite.
    \item$\pi$ is not prime. By definition, prime and composite numbers must be integers. $\pi\notin\mathbb{Z}$. Therefore, $\pi$ is neither prime nor composite.
    \item$\frac{1}{2}$ is not prime. By definition, prime and composite numbers must be integers. $\pi\notin\mathbb{Z}$. Therefore, $\pi$ is neither prime nor composite.
    \item-2 is not prime. By definition, prime numbers must be greater than 1. $-2<1$. Therefore -2 is not prime. For -2 to be composite, there must be an integer $b$ such that $1<b<-2$, which is absurd. Therefore, -2 is not composite.
    \item-1 is not prime. By definition, prime numbers must be greater than 1. $-1<1$. Therefore -2 is not prime. For -1 to be composite, there must be an integer $b$ such that $1<b<-1$, which is absurd.
\end{enumerate}
\section{Create definitions}
\subsection{Less than or equal to}
Let $x$ and $y$ be integers. We say that $x$ is \textit{less than or equal to} $y$ (denoted $x\leq y$) provided $y-x\in\mathbb{N}$.
\subsection{Less than}
Let $x$ and $y$ be integers. We say that $x$ is \textit{less than} $y$ (denoted $x<y$) provided $x\leq y$ and $x\neq y$.
\subsection{Greater than or equal to}
Let $x$ and $y$ be integers. We say that $x$ is \textit{greater than or equal to} $y$ (denoted $x\geq y$) provided that $x$ is not less than $y$.
\subsection{Greater than}
Let $x$ and $y$ be integers. We say that $x$ is \textit{greater than} $y$ (denoted $x>y$) provided that $x$ is not less than or equal to $y$.
\end{document}