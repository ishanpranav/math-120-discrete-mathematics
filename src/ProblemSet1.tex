%%%%%%%%%%%%%%%%%%%%%%%%%%%%%%%%%%%%%%%%%%%%%%%%%%%%%%%%%%%%%%%
%
% Welcome to Overleaf --- just edit your LaTeX on the left,
% and we'll compile it for you on the right. If you open the
% 'Share' menu, you can invite other users to edit at the same
% time. See www.overleaf.com/learn for more info. Enjoy!
%
%%%%%%%%%%%%%%%%%%%%%%%%%%%%%%%%%%%%%%%%%%%%%%%%%%%%%%%%%%%%%%%
\documentclass{article}
% This is a LaTeX file.  It is a text file that is compiled
% by a program called LaTeX into a pretty PDF file.  
% If you're viewing this file on CoCalc, you'll see that PDF 
% in the window to the right.
%
% The LaTeX macro language is complicated, so we have inserted
% lots of documenting comments into the file.  Comments start
% with `%' and continue to the end of the line.  In CoCalc's
% window, they are colored brownish-red.
%
% Comments prefixed with `Student:' are relevant to students.
% Skip anything else you don't understand, or ask me.
%
% set font encoding for PDFLaTeX or XeLaTeX
\usepackage{ifxetex}
\ifxetex
  \usepackage{fontspec}
\else
  \usepackage[T1]{fontenc}
  \usepackage[utf8]{inputenc}
  \usepackage{lmodern}
\fi

% Student: These lines describe some document metadata.
\title{Problem Set 1}
\author{%
% Student: change the next line to your name!
    Ishan Pranav
\\  MATH-UA 120 Discrete Mathematics
}
\date{due September 15, 2023}


\usepackage[headings=runin-fixed-nr]{exsheets}
% These make enumerates within questions start at the second ("(a)") level, rather than the first ("1.") level.
\makeatletter
    \newcommand{\stepenumdepth}{\advance\@enumdepth\@ne}
\makeatother
\SetupExSheets{
    question/pre-body-hook=\stepenumdepth,
    solution/pre-body-hook=\stepenumdepth,
}
\DeclareInstance{exsheets-heading}{runin-nn-np}{default}{
    runin = true,
    title-post-code = .\space,
    join = {
        main[r,vc]title[l,vc](0pt,0pt);
    }
}
\newif\ifshowsolutions
% Student: replace `false' with `true' to typeset your solutions.
% Otherwise they are ignored!
\showsolutionstrue
\ifshowsolutions
    \SetupExSheets{
        question/pre-hook=\itshape,
        solution/headings=runin-nn-np,
        solution/print=true,
        solution/name=Answer
    }%
    \makeatletter%
    \pretocmd{\@title}{Answers to }%
    \makeatother%
\else
    \SetupExSheets{solution/print=false}
\fi

% Bug workaround: http://tex.stackexchange.com/a/146536/1402
%\newenvironment{exercise}{}{}
\RenewQuSolPair{question}{solution}
%\let\answer\solution
%\let\endanswer\endsolution
\usepackage{manfnt}
\newcommand{\danger}{\marginpar[\hfill\dbend]{\dbend\hfill}}

\usepackage{amsmath, amsthm}
\usepackage{amsfonts}
\usepackage{enumerate}
\usepackage{siunitx}
\DeclareSIUnit\pound{lb}
\usepackage{hyperref}
\newtheorem*{theorem}{Theorem}
\newtheorem*{claim}{Claim}
\theoremstyle{definition}
\newtheorem*{definition}{Definition}
% We are creating a command "\xor".
\newcommand{\xor}{\underline{\lor}}
% This is the beginning of the part of the file that describes
% the text of the document.
% That's why it says `\begin{document}' below. :-)
\begin{document}
\maketitle



These are to be written up and turned in to Gradescope.\\



\ifshowsolutions
    \SetupExSheets{solution/print=true}
\else
    \danger
 \underline{ \LaTeX  Instructions:}  You can view the source (\texttt{.tex}) file to get some more examples of \LaTeX{} code.  I have commented in the source file in places where new \LaTeX{} constructions are used.
  
  Remember to change \verb|\showsolutionsfalse| to \verb|\showsolutionstrue|
    in the document's preamble 
    (between \verb|\documentclass{article}| and \verb|\begin{document}|)
\fi

\section*{Assigned Problems}


\begin{question}
    % Notice the use of the enumerate environment
    % to make a numbered list.  Each item is marked
    % by the \item command.
    %
    % Also \emph = emphasize, usually in italics.
    Let the following statements be given. 
       \begin{definition}
          A triangle is \emph{scalene} if all of its sides have different lengths.
       \end{definition}
       \begin{theorem}
          A triangle is scalene if it is a right triangle that is not isosceles.
       \end{theorem}
    Suppose $\Delta ABC$ is a scalene triangle. 
    Which of the following conclusions are valid based only on the information given above? 
    Why or why not?
    \begin{enumerate}
        \item All of the sides of $\Delta ABC$ have different lengths.
        \item $\Delta ABC$ is a right triangle that is not isosceles.
    \end{enumerate}
\end{question}
% Student: put your answer between the next two lines.
\begin{solution}
\end{solution}


\begin{question}
   Without changing their meanings, convert each of the following sentences into a sentence of the form ``For all ... $x$, if $x$ ... , then .'', regardless of whether the statement is true or false.
    \begin{enumerate}
        \item Every prime greater than 2 is odd.
        \item Three consecutive odd integers greater than 3 cannot all be prime.
        \item An integer is divisible by 8 only if it is divisible by 4.
        \item You fail only if you stop writing.
        \item People will generally accept facts as truth only if the facts agree with what they already believe.
    \end{enumerate}
\end{question}
% Student: put your answer between the next two lines.
\begin{solution}
\end{solution}


\begin{question}
   The following claim and its proof is poorly written. They are both missing some crucial information. 
   Please revise both the claim and proof so that any student in this course will understand it. Explain your reasoning.
      \begin{claim}
       If $x^2\neq 0$, then $x^2>0$.
      \end{claim}
      \begin{proof}
       If $x>0$, then $x^=xx>0$. If $x<0$, then $-x>0$, so $(-x)(-x)>0$, i.e., $x^2>0$.
      \end{proof}
\end{question}
% Student: put your answer between the next two lines.
\begin{solution}
\end{solution}


\begin{question}
    Consider the following definition of the ``$\triangleleft$'' symbol.
	\begin{definition}
	 Let $x$ and $y$ be integers. Write $x\triangleleft y$ if $3x+5y=7k$ for some integer $k$.
	\end{definition}
        \begin{enumerate}
           \item Show that $1\triangleleft 5$, $3\triangleleft 1$, and $0\triangleleft 7$.
           \item Prove that if $a\triangleleft b$ and $c\triangleleft d$, then $(a+c) \triangleleft (b+d)$.
        \end{enumerate}
\end{question}
% Student: put your answer between the next two lines.
\begin{solution}
\end{solution}

\begin{question}
    Show that an integer $n$ is odd if and only if $2n+2$ is divisible by 4.
\end{question}
% Student: put your answer between the next two lines.
\begin{solution}
\end{solution}


\end{document}
