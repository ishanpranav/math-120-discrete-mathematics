\documentclass{article}
\usepackage{ifxetex}
\ifxetex
  \usepackage{fontspec}
\else
  \usepackage[T1]{fontenc}
  \usepackage[utf8]{inputenc}
  \usepackage{lmodern}
\fi
\title{Problem Set 1}
\author{%
    Ishan Pranav
\\  MATH-UA 120 Discrete Mathematics
}
\date{due September 15, 2023}
\usepackage[headings=runin-fixed-nr]{exsheets}
\makeatletter
    \newcommand{\stepenumdepth}{\advance\@enumdepth\@ne}
\makeatother
\SetupExSheets{
    question/pre-body-hook=\stepenumdepth,
    solution/pre-body-hook=\stepenumdepth,
}
\DeclareInstance{exsheets-heading}{runin-nn-np}{default}{
    runin = true,
    title-post-code = .\space,
    join = {
        main[r,vc]title[l,vc](0pt,0pt);
    }
}
\newif\ifshowsolutions
\showsolutionstrue
\ifshowsolutions
    \SetupExSheets{
        question/pre-hook=\itshape,
        solution/headings=runin-nn-np,
        solution/print=true,
        solution/name=Answer
    }%
    \makeatletter%
    \pretocmd{\@title}{Answers to }%
    \makeatother%
\else
    \SetupExSheets{solution/print=false}
\fi
% Bug workaround: http://tex.stackexchange.com/a/146536/1402
%\newenvironment{exercise}{}{}
\RenewQuSolPair{question}{solution}
\usepackage{manfnt}
\newcommand{\danger}{\marginpar[\hfill\dbend]{\dbend\hfill}}
\usepackage{amsmath, amsthm}
\usepackage{amsfonts}
\usepackage{enumerate}
\usepackage{siunitx}
\DeclareSIUnit\pound{lb}
\usepackage{hyperref}
\newtheorem*{theorem}{Theorem}
\newtheorem*{claim}{Claim}
\theoremstyle{definition}
\newtheorem*{definition}{Definition}
\newcommand{\xor}{\underline{\lor}}
\begin{document}
\maketitle
These are to be written up and turned in to Gradescope.\\
\ifshowsolutions
    \SetupExSheets{solution/print=true}
\else
    \danger
 \underline{ \LaTeX  Instructions:}  You can view the source (\texttt{.tex}) file to get some more examples of \LaTeX{} code.  I have commented in the source file in places where new \LaTeX{} constructions are used.
  
  Remember to change \verb|\showsolutionsfalse| to \verb|\showsolutionstrue|
    in the document's preamble 
    (between \verb|\documentclass{article}| and \verb|\begin{document}|)
\fi
\section*{Assigned Problems}
\begin{question}
    Let the following statements be given. 
       \begin{definition}
          A triangle is \emph{scalene} if all of its sides have different lengths.
       \end{definition}
       \begin{theorem}
          A triangle is scalene if it is a right triangle that is not isosceles.
       \end{theorem}
    Suppose $\triangle ABC$ is a scalene triangle. 
    Which of the following conclusions are valid based only on the information given above? 
    Why or why not?
    \begin{enumerate}
        \item All of the sides of $\triangle ABC$ have different lengths.
        \item $\triangle ABC$ is a right triangle that is not isosceles.
    \end{enumerate}
\end{question}
\begin{solution}
Statement (a) is valid based only on the information given above. It is given that $\triangle ABC$ is a scalene triangle. By the definition of a scalene triangle (Definition), all three of the triangle's sides have different lengths.

Statement (b) is not necessarily valid based on the information given above. Although a triangle is scalene if it is a right triangle that is not isosceles (Theorem), neither the above definition nor the above theorem establish that a right triangle that is not isosceles is necessarily scalene. The theorem is a positive statement and must be true. However, without a demonstration, one may not assume that its converse is true.
\end{solution}
\begin{question}
   Without changing their meanings, convert each of the following sentences into a sentence of the form ``For all ... $x$, if $x$ ... , then .'', regardless of whether the statement is true or false.
    \begin{enumerate}
        \item Every prime greater than 2 is odd.
        \item Three consecutive odd integers greater than 3 cannot all be prime.
        \item An integer is divisible by 8 only if it is divisible by 4.
        \item You fail only if you stop writing.
        \item People will generally accept facts as truth only if the facts agree with what they already believe.
    \end{enumerate}
\end{question}
\begin{solution}
\begin{enumerate}
    \item For all integers $x$, if $x$ is prime and $x>2$, then $x$ is odd.
    \item For all integers $x$, if $x$ is odd and $x>3$, then $x$ is not prime, $x+2$ is not prime, or $x+4$ is not prime.
    \item For all integers $x$, if $x\,|\,8$, then $x\,|\,4$.
    \item For all persons $x$, if $x$ fails, then $x$ stops writing.
    \item For all persons $x$, if $x$ generally accepts facts as truth, then the facts agree with what $x$ already believes.
\end{enumerate}
\end{solution}
\begin{question}
   The following claim and its proof are poorly written. They are both missing some crucial information. 
   Please revise both the claim and proof so that any student in this course will understand it. Explain your reasoning.
      \begin{claim}
       If $x^2\neq 0$, then $x^2>0$.
      \end{claim}\begin{proof}
       If $x>0$, then $x^2=xx>0$. If $x<0$, then $-x>0$, so $(-x)(-x)>0$, i.e., $x^2>0$.
      \end{proof}
\end{question}
\begin{solution}
\begin{claim}
Let $x\in\mathbb{N}$. If $x^2\neq 0$, then $x^2>0$.
\end{claim}\begin{proof}
Since $x\in\mathbb{N}$, $x<0$, $x=0$, or $x>0$.

Suppose $x<0$. Let $a=-x\in\mathbb{N}$. Observe
\begin{align*}
x^2
&=x\cdot x\\
&=(-a)(-a)\\
&=a^2.
\end{align*}
Note $a\in\mathbb{N}$, so $a^2\in\mathbb{Z}\geq 0$. Transitively, $x^2=a^2$, so $x^2\geq 0$. According to the hypothesis, $x^2\neq 0$, so $x^2>0$. Therefore, if $x<0$, then $x^2>0$.

Suppose $x=0$. Then $x^2=x\cdot x=0$. But $x^2\neq 0$ according to the hypothesis. Therefore $x\neq 0$.

Suppose $x>0$. Let $b=x\in\mathbb{N}$. Observe $x^2=b^2$. Note $b\in\mathbb{N}$, so $b^2\in\mathbb{Z}\geq 0$. Transitively, $x^2=b^2$, so $x^2\geq 0$.  According to the hypothesis, $x^2\neq 0$, so $x^2>0$. Therefore, if $x>0$, then $x^2>0$.

In all (both) cases which satisfy the hypothesis, $x^2>0$. We conclude that if $x^2\neq 0$, then $x^2>0$.
\end{proof}
\end{solution}
\begin{question}
    Consider the following definition of the ``$\triangleleft$'' symbol.
	\begin{definition}
	 Let $x$ and $y$ be integers. Write $x\triangleleft y$ if $3x+5y=7k$ for some integer $k$.
	\end{definition}
        \begin{enumerate}
           \item Show that $1\triangleleft 5$, $3\triangleleft 1$, and $0\triangleleft 7$.
           \item Prove that if $a\triangleleft b$ and $c\triangleleft d$, then $(a+c) \triangleleft (b+d)$.
        \end{enumerate}
\end{question}
\begin{solution}
\begin{enumerate}
\item We want to find $k\in\mathbb{Z}$ such that $3(1)+5(5)=7k$. Let $k=4$. Then $3(1)+5(5)=28=7(4)$. There exists $k\in\mathbb{Z}$ such that $3(1)+5(5)=7k$. Therefore, $1\triangleleft 5$.

We want to find $k\in\mathbb{Z}$ such that $3(3)+5(1)=7k$. Let $k=2$. Then $3(3)+5(1)=14=7(2)$. There exists $k\in\mathbb{Z}$ such that $3(3)+5(1)=7k$. Therefore, $3\triangleleft 1$.

We want to find $k\in\mathbb{Z}$ such that $3(0)+5(7)=7k$. Let $k=5$. Then $3(0)+5(7)=35=7(5)$. There exists $k\in\mathbb{Z}$ such that $3(0)+5(7)=7k$. Therefore, $0\triangleleft 7$.
\item Consider the following proof.\begin{claim}
Let $a,b,c,d\in\mathbb{Z}$. If $a\triangleleft b$ and $c\triangleleft d$, then $(a+c)\triangleleft(b+d)$.
\end{claim}\begin{proof}
Since $a\triangleleft b$, there exists $x\in\mathbb{Z}$ such that $3a+5b=7x$. Since $c\triangleleft d$, there exists $y\in\mathbb{Z}$ such that $3c+5d=7y$. Observe
\begin{align*}
(3a+5b)+(3c+5d)&=(7x)+(7y)\\
3a+5b+3c+5d&=7x+7y\\
3a+3c+5b+5d&=7x+7y\\
3(a+c)+5(b+d)&=7(x+y).
\end{align*}
Let $k=(x+y)\in\mathbb{Z}$. There exists $k\in\mathbb{Z}$ such that $3(a+c)+5(b+d)=7k$. Therefore, $(a+c)\triangleleft(b+d)$. We conclude that if $a\triangleleft b$ and $c\triangleleft d$, then $(a+c)\triangleleft(b+d)$.
\end{proof}
\end{enumerate}
\end{solution}
\begin{question}
    Show that an integer $n$ is odd if and only if $2n+2$ is divisible by 4.
\end{question}
\begin{solution}
\begin{claim}
Let $n\in\mathbb{Z}$; $n$ is odd if and only if $4\,|\,(2n+2)$.
\end{claim}\begin{proof}
Let $n\in\mathbb{Z}$.

First, we will prove that if $4\,|\,(2n+2)$, then $n$ is odd. Suppose $4\,|\,(2n+2)$. Since $4\,|\,(2n+2)$, there exists $a\in\mathbb{Z}$ such that $(2n+2)=4a$. Observe
\begin{align*}
(2n+2)&=4a\\
2n&=4a-2\\
n&=2a-1\\
n&=2(a-1)+1.
\end{align*}
Let $b=(a-1)$. There exists $b\in\mathbb{Z}$ such that $n=2b+1$. Thus, $n$ is odd. Therefore, if $4\,|\,(2n+2)$, then $n$ is odd.

Next, we will prove that if $n$ is odd, then $4\,|\,(2n+2)$. Suppose $n$ is odd. Since $n$ is odd, there exists $c\in\mathbb{Z}$ such that $n=2c+1$. Observe
\begin{align*}
n=2c+1\\
2n=4c+2\\
2n+2=4c+4\\
(2n+2)=4(c+1).
\end{align*}
Let $d=c+1\in\mathbb{Z}$. There exists $d\in\mathbb{Z}$ such that $(2n+2)=4d$. Thus, $(2n+2)\,|\,4$. Therefore, if $n$ is odd then $4\,|\,(2n+2)$.

We conclude that $n$ is odd if and only if $4\,|\,(2n+2)$.
\end{proof}
\end{solution}
\end{document}