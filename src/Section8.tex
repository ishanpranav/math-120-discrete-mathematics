\documentclass[12pt]{article}
\usepackage[english]{babel}
\usepackage[letterpaper,top=2cm,bottom=2cm,left=3cm,right=3cm,marginparwidth=1.75cm]{geometry}
\usepackage{amsmath}
\usepackage{amsfonts}
\usepackage{graphicx}
\title{MATH-UA 120 Section 8}
\author{Ishan Pranav}
\date{September 16, 2023}
\DeclareMathOperator{\len}{len}
\begin{document}
\maketitle
\section*{List}
A \textit{list} or \textit{tuple} is an ordered sequence of objects. Elements in a list might be repeated.
\section*{Length}
The number of elements in a list $l$ is called its \textit{length} and is denoted $\len(l)$.
\section*{Ordered pair}
A list of length two is called an \textit{ordered pair}.
\section*{Empty list}
A list of length zero is called the \textit{empty list} and is denoted $(\,)$.
\section*{Equal}
Two lists are \textit{equal} provided they have the same length, and elements in the corresponding positions on the two lists are equal. Lists $(a,b,c)$ and $(x,y,z)$ are equal if and only if $a=x,b=y$.
\section*{Concatenation}
Suppose $A$ and $B$ are lists. Their \textit{concatenation} is the new list formed by listing first the elements of $A$ and then the elements $B$. The concatenation of the lists $(1,2,1)$ and $(1,3,5)$ is the list $(1,2,1,1,3,5)$.
\section*{Multiplication principle}
Consider two-element lists for which there are $n$ choices for the first element, and for each choice of the first element, there are $m$ choices for the second element. Then the number of such lists is $nm$. 
\section*{Falling factorial}
The quantity
\[(x)_n=\prod_{k=1}^n{(x-k+1)}=\prod_{k=0}^{n-1}{(x-k)}.\]
is called a \textit{falling factorial}, \textit{descending factorial}, \textit{falling sequential product}, or \textit{lower factorial}. The quantity $(r)_n$ is equal to the number of $r$-permutations from a set of $n$ items, called a \textit{partial permutation} and denoted ${}_nP_r=\frac{n!}{(n-r)!}$.
\section*{List-counting theorem}
The number of lists of length $k$ whose elements are chosen from a pool of $n$ possible elements is $n^k$ if repetitions are permitted and $(n)_k$ if repetitions are forbidden.
\end{document}